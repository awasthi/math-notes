%
% Notes on Mathematics
% John Peloquin
%
% Algebra
% Vector Spaces
% Eigenvalues and Eigenvectors
%
\section{Eigenvalues and Eigenvectors}
\subsection*{Definitions}
\begin{defn}
If \(T\in\Hom(V)\), a subspace~\(U\) of~\(V\) is \emph{invariant} under~\(T\) if \(Tu\in U\) for all \(u\in U\).
\end{defn}

\begin{defn}
If \(T\in\Hom(V)\), \(v\in V\), and \(\lambda\in\F\), \(v\)~is an \emph{eigenvector} of~\(T\) corresponding to~\(\lambda\) if \(Tv=\lambda v\). If there is a nonzero eigenvector of~\(T\) corresponding to~\(\lambda\), then \(\lambda\)~is called an \emph{eigenvalue} of~\(T\).
\end{defn}
\begin{rmk}
Note eigenvalues and eigenvectors can be zero, but corresponding to an eigenvalue there must always be a nonzero eigenvector.
\end{rmk}

\begin{defn}
If \(T\in\Hom(V)\) and \(p(z)\)~is a polynomial over~\(\F\) of the form
\[p(z)=a_0+a_1z+\cdots+a_nz^n\]
where \(a_1,\ldots,a_n\in\F\), then the \emph{polynomial operator}~\(p(T)\) is
\[p(T)=a_0I+a_1T+\cdots+a_nT^n\]
\end{defn}

\begin{defn}
Let \(M\)~be a square matrix. \(M\)~is \emph{diagonal} if \(M_{ij}=0\) whenever \(i\ne j\). \(M\)~is \emph{upper-triangular} if \(M_{ij}=0\) whenever \(i>j\). \(M\)~is \emph{block upper-triangular} if
\[M=\left[\begin{matrix}
M_1&&*\\
&\ddots&\\
0&&M_n
\end{matrix}\right]\]
where each~\(M_i\) is a square matrix called a \emph{block}.
\end{defn}

\subsection*{Theorems}
\begin{thm}[Characterizations of eigenvalue]
If \(T\in\Hom(V)\) and \(\lambda\in\F\), the following are equivalent:
\begin{enumerate}[itemsep=0pt]
\item[(a)] \(\lambda\)~is an eigenvalue of~\(T\)
\item[(b)] \(\ker(T-\lambda I)\) is nonzero
\item[(c)] \(T-\lambda I\) is not injective
\item[(d)] \(T-\lambda I\) is not surjective
\item[(e)] \(T-\lambda I\) is not invertible
\end{enumerate}
\end{thm}

\begin{thm}[Eigenvalues and invariant subspaces]
A linear operator on a vector space has an eigenvalue iff it has a one-dimensional invariant subspace.
\end{thm}

\begin{thm}[Linear independence of eigenvectors]
Nonzero eigenvectors of a linear operator corresponding to distinct eigenvalues are linearly independent.
\end{thm}
\begin{proof}[Proof idea]
By contradiction, looking at a redundant vector.

Let \(T\in\Hom(V)\), \(\lambda_1,\ldots,\lambda_n\) distinct eigenvalues of~\(T\), and \(v_1,\ldots,v_n\) corresponding nonzero eigenvectors. If \((v_1,\ldots,v_n)\) is linearly dependent, fix \(k>1\) least such that \(v_k\in\spn(v_1,\ldots,v_{k-1})\). So \((v_1,\ldots,v_{k-1})\) is linearly independent but 
\[v_k=a_1v_1+\cdots+a_{k-1}v_{k-1}\]
for some \(a_1,\ldots,a_{k-1}\in\F\) not all zero. Applying~\(T\) to both sides gives
\[\lambda_kv_k=a_1\lambda_1v_1+\cdots+a_{k-1}\lambda_{k-1}v_{k-1}\]
But \(\lambda_kv_k=a_1\lambda_kv_1+\cdots+a_{k-1}\lambda_kv_k\), so
\[0=a_1(\lambda_1-\lambda_k)v_1+\cdots+a_{k-1}(\lambda_{k-1}-\lambda_k)v_{k-1}\]
Therefore \(\lambda_i=\lambda_k\) for some \(i\in\{1,\ldots,k-1\}\), contradicting distinctness.
\end{proof}

\begin{cor}
\(T\)~has at most~\(\dim V\) distinct eigenvalues.
\end{cor}

\begin{cor}
If \(T\)~has \(\dim V\)~distinct eigenvalues, then \(T\)~has a diagonal matrix with respect to a basis of eigenvectors.
\end{cor}

\begin{thm}[Polynomial operators]
If \(p,q\in\F[x]\), \(T\)~is a linear operator, and \(c\in\F\),
\begin{enumerate}[itemsep=0pt]
\item[(a)] \((p+q)(T)=p(T)+q(T)\)
\item[(b)] \((cp)(T)=cp(T)\)
\item[(c)] \((pq)(T)=p(T)q(T)\)
\end{enumerate}
\end{thm}
\begin{proof}[Proof idea]
For (a)~and~(b), by substitution.

For~(c), by substition and induction on~\(n=\deg q\). If \(n\le0\), the result is trivial. If \(n>0\), let \(c\)~be the leading coefficient of~\(q(z)\). Then
\[(pq)(z)=p(z)q(z)=p(z)[q(z)-cz^n]+cp(z)z^n\]
Now \(\deg[q(z)-cz^n]<n\), so by linearity and induction we have
\begin{align*}
(pq)(T)&=[p(z)[q(z)-cz^n]](T)+[cp(z)z^n](T)\\
	&=p(T)[q(T)-cT^n]+c[p(z)z^n](T)\\
	&=p(T)q(T)-c[p(T)T^n-[p(z)z^n](T)]
\end{align*}
It is easy to verify \(p(T)T^n=[p(z)z^n](T)\), so \((pq)(T)=p(T)q(T)\).
\end{proof}
\begin{cor}
Polynomial operator substitution \(p\to p(T)\) is linear.
\end{cor}
\begin{cor}
Polynomial operators over~\(T\) commute.
\end{cor}

\begin{thm}[Eigenvalues over~\(\C\)]
Every linear operator on a nonzero, finite-dimensional complex vector space has an eigenvalue.
\end{thm}
\begin{proof}[Proof idea]
By factoring a complex polynomial operator.

Let \(V\)~be such a vector space and \(T\in\Hom(V)\). Set \(n=\dim V\) and fix \(v\in V\), \(v\ne0\). Then the list \((v,Tv,\ldots,T^nv)\) is linearly dependent, so there exist \(a_0,\ldots,a_n\in\C\) such that
\[a_0v+a_1Tv+\cdots+a_nT^nv=0\]
Set \(p(z)=a_0+a_1z+\cdots+a_nz^n\). By polynomial factorization over~\(\C\), there exist values \(c,\lambda_1,\ldots,\lambda_m\in\C\) with \(c\ne0\) and \(m\ge1\) such that \(p(z)=c(z-\lambda_1)\cdots(z-\lambda_m)\). So
\begin{align*}
0&=a_0v+a_1Tv+\cdots+a_nT^nv\\
	&=(a_0I+a_1T+\cdots+a_nT^n)v\\
	&=c(T-\lambda_1I)\cdots(T-\lambda_mI)v
\end{align*}
Therefore some \((T-\lambda_iI)\) is not injective, and \(\lambda_i\)~is an eigenvalue of~\(T\).
\end{proof}
\begin{app}
Simplifying matrices.
\end{app}

\begin{cor}[Upper-triangular matrices over~\(\C\)]
Every linear operator on a nonzero, finite-dimensional complex vector space has an upper-triangular matrix with respect to some basis.
\end{cor}
\begin{proof}[Proof idea]
By induction on the dimension of the space, using an eigenvalue in the induction step.

Let \(V\)~be such a space and \(T\in\Hom(V)\). Fix an eigenvalue~\(\lambda\) of~\(T\). Then \(U=\range(T-\lambda I)\) is invariant under~\(T\) and \(\dim U<\dim V\), so there is a basis of~\(U\) with respect to which \(T|_U\)~has an upper-triangular matrix. Extend this to a basis of~\(V\) with respect to which \(T\)~has an upper-triangular matrix.
\end{proof}

\begin{thm}[Upper-triangular matrices]
If a linear operator has an upper-triangular matrix with respect to some basis, then the operator is invertible iff there are no zeros along the diagonal of the matrix.
\end{thm}
\begin{proof}[Proof idea]
By rank nullity.

Let \(T\in\Hom(V)\) and suppose \((v_1,\ldots,v_n)\) is a basis of~\(V\) such that \(\mat(T,(v_1,\ldots,v_n))\) is upper-triangular. If the \(k\)-th value along the diagonal is zero, \(T\)~maps \(\spn(v_1,\ldots,v_k)\) into \(\spn(v_1,\ldots,v_{k-1})\) of smaller dimension, so \(T\)~is not injective and not invertible. Conversely, if \(T\)~is not invertible, then \(T\)~is not surjective so there exists \(k\)~least such that \(v_k\not\in\range T|_{\spn(v_1,\ldots,v_k)}\). It is easy to see that the \(k\)-th value along the diagonal must be zero.
\end{proof}

\begin{cor}
The eigenvalues of the operator are the values along the diagonal.
\end{cor}
\begin{proof}[Proof idea]
Since for any \(\lambda\in\F\),
\begin{equation*}
\mat(T-\lambda I,(v_1,\ldots,v_n))=\left[\begin{matrix}
\lambda_1-\lambda&&*\\
&\ddots&\\
0&&\lambda_n-\lambda
\end{matrix}\right]\qedhere
\end{equation*}
\end{proof}

\begin{thm}[Eigenvalues over~\(\R\)]
Every linear operator on a nonzero, finite-dimensional real vector space has an invariant subspace of dimension \(1\)~or~\(2\).
\end{thm}
\begin{proof}[Proof idea]
By factoring a real polynomial operator.

Note an invariant subspace of dimension~\(1\) corresponds to a linear factor, and an invariant subspace of dimension~\(2\) to an irreducible quadratic factor.
\end{proof}
\begin{app}
Simplifying matrices.
\end{app}

\begin{rmk}
The differences between complex and real operators stem largely from the differences between complex and real polynomial factorization.
\end{rmk}

\begin{cor}
Every linear operator on a real vector space of odd dimension has an eigenvalue.
\end{cor}
\begin{proof}[Proof idea]
By induction on the dimension of the space, using an invariant subspace of dimension \(1\)~or~\(2\) in the induction step.

Let \(V\)~be such a vector space and \(T\in\Hom(V)\). Let \(U\)~be an invariant subspace of dimension \(1\)~or~\(2\). If \(U\)~has dimension~\(1\), \(T\)~has an eigenvalue. If \(U\)~has dimension~\(2\), write \(V=U\directsum W\). Then \(W\)~has odd dimension and is invariant under~\(P_{W,U}T|_W\), so by induction \(P_{W,U}T|_W\)~has an eigenvalue, which is also an eigenvalue of~\(T\).
\end{proof}

\begin{cor}[Block upper-triangular matrices over~\(\R\)]
Every linear operator on a nonzero, finite-dimensional real vector space has a block upper-trianguar matrix with respect to some basis whose blocks are either \(1\)-by-\(1\) or \(2\)-by-\(2\) with no eigenvalues.
\end{cor}
\begin{proof}[Proof idea]
By induction on the dimension of the space, using an invariant subspace of dimension \(1\)~or~\(2\) in the induction step.
\end{proof}

\begin{thm}[Diagonalization]
Let \(V\)~be finite-dimensional, \(T\in\Hom(V)\), and \(\lambda_1,\ldots,\lambda_m\) the distinct eigenvalues of~\(T\). The following are equivalent:
\begin{enumerate}[itemsep=0pt]
\item[(a)] \(T\)~has a diagonal matrix with respect to some basis of~\(V\)
\item[(b)] \(V\)~has a basis consisting of eigenvectors of~\(T\)
\item[(c)] \(T\)~has one-dimensional invariant subspaces \(U_1,\ldots,U_n\) such that
\[V=U_1\directsum\cdots\directsum U_n\]
\item[(d)] \(V=\ker(T-\lambda_1I)\directsum\cdots\directsum\ker(T-\lambda_m I)\)
\item[(e)] \(\dim V=\dim\ker(T-\lambda_1I)+\cdots+\dim\ker(T-\lambda_m I)\)
\end{enumerate}
\end{thm}

\subsection*{Techniques}
\begin{itemize}[itemsep=0pt]
\item Applying polynomial theory to polynomial operators.
\item Induction on dimension.
\item Simplifying matrices of operators for ease of analysis and computation.
\item Using rank nullity to analyze the behavior of operators.
\item Translating between equality and nullity.
\end{itemize}
