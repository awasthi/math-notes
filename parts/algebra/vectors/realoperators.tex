%
% Notes on Mathematics
% John Peloquin
%
% Algebra
% Vector Spaces
% Operators on Real Vector Spaces
%
\section{Operators on Real Vector Spaces}
\subsection*{Definitions}
\begin{defn}
If \(V\)~is real and \(T\in\Hom(V)\), \((\alpha,\beta)\in\R^2\) is an \emph{eigenpair} of~\(T\) if \(\alpha^2<4\beta\) and \(T^2+\alpha T+\beta I\) is not injective.
\end{defn}

\begin{defn}
If \(V\)~is real, \(T\in\Hom(V)\), and \(\mat(T)\)~is block upper-triangular of the form
\[\mat(T)=\left[\begin{matrix}
A_1&&*\\
&\ddots&\\
0&&A_m
\end{matrix}\right]\]
where each~\(A_i\) is either a \(1\)-by-\(1\) block of the form~\([\lambda_i]\) or a \(2\)-by-\(2\) block of the form \(\left[\begin{matrix}a_i&c_i\\b_i&d_i\end{matrix}\right]\) with no eigenvalues, the \emph{characteristic polynomial} of~\(T\) is
\[p(x)=p_1(x)\cdots p_m(x)\]
where
\[p_i(x)=\begin{cases}
x-\lambda_i&\text{if }A_i\text{ is 1-by-1}\\
(x-a_i)(x-d_i)-b_ic_i&\text{if }A_i\text{ is 2-by-2}
\end{cases}\]
\end{defn}

\begin{defn}
The multiplicity of an eigenpair \((\alpha,\beta)\) of~\(T\) is \(\dim\ker(T^2+\alpha T+\beta I)^{\dim V}/2\).
\end{defn}

\subsection*{Theorems}
\begin{rmk}
Proofs of results here are analogous to the proofs of corresponding results for complex vector spaces, and so are omitted.
\end{rmk}

\begin{thm}[Eigenvalue and eigenpair multiplicity over~\(\R\)]
If \(V\)~is finite-dimensional and real and \(T\in\Hom(V)\), then for any block upper-triangular matrix for~\(T\) consisting of \(1\)-by-\(1\) blocks and \(2\)-by-\(2\) blocks with no eigenvalues,
\begin{enumerate}[itemsep=0pt]
\item[(a)] For \(\lambda\in\R\), \([\lambda]\)~appears as a \(1\)-by-\(1\) block \(\dim\ker(T-\lambda I)^{\dim V}\) times.
\item[(b)] For \(\alpha,\beta\in\R\) with \(\alpha^2<4\beta\), \(x^2+\alpha x+\beta\) appears as the characteristic polynomial of a \(2\)-by-\(2\) block \(\dim\ker(T^2+\alpha T+\beta I)^{\dim V}/2\) times.
\end{enumerate}
\end{thm}
\begin{app}
Multiplicity.
\end{app}
\begin{cor}
\[\dim V=\sum\text{multiplicities of eigenvalues of~\(T\)}+2\sum\text{multiplicities of eigenpairs of~\(T\)}\]
\end{cor}
\begin{rmk}
By the theorem, there is an invariance with respect to the characteristic polynomials of blocks occurring in block upper-triangular representations of real operators, but this is not true in general with respect to the blocks themselves.
\end{rmk}

\begin{thm}[Generalized eigenspace and eigenpair space decomposition over~\(\R\)] 
If \(V\)~is finite-dimensional and real, \(T\in\Hom(V)\), \(\lambda_1,\ldots,\lambda_m\in\R\) are the distinct eigenvalues of~\(T\) with corresponding generalized eigenspaces \(U_1,\ldots,U_m\), and \((\alpha_1,\beta_1)\),\ldots,\((\alpha_M,\beta_M)\) are the distinct eigenpairs of~\(T\) with corresponding eigenpair spaces \(V_1,\ldots,V_M\), then
\begin{enumerate}[itemsep=0pt]
\item[(a)] \(V=U_1\directsum\cdots\directsum U_m\directsum V_1\directsum\cdots\directsum V_M\).
\item[(b)] \(U_i\) is invariant under~\(T\) for \(i\in\{1,\ldots,m\}\).
\item[(c)] \(V_j\)~is invariant under~\(T\) for \(j\in\{1,\ldots,M\}\).
\item[(d)] \((T-\lambda_i I)|_{U_i}\)~is nilpotent on~\(U_i\) for \(i\in\{1,\ldots,m\}\).
\item[(e)] \((T^2+\alpha_jT+\beta_j I)|_{U_j}\)~is nilpotent on~\(V_j\) for \(j\in\{1,\ldots,M\}\).
\end{enumerate}
\end{thm}

\begin{thm}[Cayley-Hamilton]
If \(V\)~is finite-dimensional and real, \(T\in\Hom(V)\), and \(p\)~is the characteristic polynomial of~\(T\), then \(p(T)=0\).
\end{thm}

\subsection*{Techniques}
\begin{itemize}[itemsep=0pt]
\item Simplifying matrices of operators for ease of analysis and computation using the generalized eigenspace and eigenpair space decomposition.
\end{itemize}
