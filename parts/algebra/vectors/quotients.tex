%
% Notes on Mathematics
% John Peloquin
%
% Algebra
% Vector Spaces
% Quotients
%
\section{Quotients}
\subsection*{Definitions}
\begin{defn}
If \(U\)~is a subspace of~\(V\), then for \(v\in V\), the set
\[v+U=\{\,v+u\mid u\in U\,\}\]
is called the \emph{coset} of~\(U\) containing~\(v\).
\end{defn}
\begin{defn}
If \(U\)~is a subspace of~\(V\), then the \emph{quotient space} of~\(V\) modulo~\(U\), denoted~\(V/U\), is the set of cosets of~\(U\) in~\(V\)
\[V/U=\{\,v+U\mid v\in V\,\}\]
with the following laws of composition:
\begin{align*}
(v+U)+(w+U)&=(v+w)+U\\
\alpha(v+U)&=\alpha v+U
\end{align*}
for all \(v,w\in V\) and \(\alpha\in\F\).
\end{defn}
\begin{defn}
If \(U\)~is a subspace of~\(V\), the \emph{canonical map} from~\(V\) to~\(V/U\) is the map \(\pi:V\to V/U\) mapping \(v\mapsto(v+U)\).
\end{defn}
\subsection*{Theorems}
\begin{thm}[Quotient space]
If \(U\)~is a subspace of~\(V\), then the quotient space~\(V/U\) forms a vector space.
\end{thm}

\begin{thm}[First isomorphism theorem]
If \(T\in\Hom(V,W)\) is surjective and \(U=\ker T\), the map \(\overline{T}:V/U\to W\) mapping \((v+U)\mapsto Tv\) witnesses an isomorphism \(V/U\iso W\). Moreover, \(T=\overline{T}\pi\), where \(\pi\)~is the canonical map.
\end{thm}
\begin{proof}
By direct computation,
\begin{align*}
v+U=w+U&\iff(v-w)+U=U\\
        &\iff(v-w)\in U\\
        &\iff T(v-w)=0\\
        &\iff Tv-Tw=0\\
        &\iff Tv=Tw
\end{align*}
for all \(v,w\in V\), so \(\overline{T}\)~is well-defined and injective. Linearity and surjectivity of~\(\overline{T}\) are induced from~\(T\), so \(\overline{T}\)~is an isomorphism. Finally,
\[(\overline{T}\pi)v=\overline{T}(\pi(v))=\overline{T}(v+U)=Tv\qedhere\]
\end{proof}
\begin{app}
Induction on the order of finite vector spaces.
\end{app}
\subsection*{Techniques}
\begin{itemize}[itemsep=0pt]
\item Induction on the order of finite vector spaces.
\end{itemize}
