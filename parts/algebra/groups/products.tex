%
% Notes on Mathematics
% John Peloquin
%
% Algebra
% Groups
% Direct Products and Semidirect Products
%
\section{Direct Products and Semidirect Products}
\subsection*{Theorems}
\begin{thm}[Fundamental theorem of finite abelian groups]
If \(G\)~is an abelian group with \(\ord{G}=p_1^{a_1}\cdots p_k^{a_k}\), where \(p_1,\ldots,p_k\) are pairwise distinct primes and \(a_1,\ldots,a_k\ge1\), there exist groups \(A_1,\ldots,A_k\) such that
\begin{enumerate}[itemsep=0pt]
\item[(a)] \(G\iso\prod_{i=1}^k A_i\)
\item[(b)] For each factor~\(A_i\), \(\ord{A_i}=p_i^{a_i}\) and there exist \(b_1\ge\cdots\ge b_m\ge 1\) with \(a_i=\sum_{j=1}^m b_j\) such that \(A_i\iso\prod_{j=1}^m Z_{p_i^{b_j}}\).
\end{enumerate}
Moreover, this factorization is unique in the sense that if \(G\iso\prod_{i=1}^r B_i\) with \(\ord{B_i}=p_i^{a_i}\) for all~\(i\), then \(A_i\iso B_i\) for all~\(i\).
\end{thm}
\begin{proof}[Proof idea]
By the structure theorem for finitely generated modules over principal ideal domains, since finite abelian groups are just finitely generated \(\Z\)-modules.

Alternately, for existence, by induction. Since a finite abelian group is a direct product of Sylow subgroups,\footnote{See characterizations of finite nilpotence below.} assume without loss of generality that \(G\)~is a \(p\)-group. First argue that an elementary abelian \(p\)-group~\(E\) can be factored as \(E=M\times\groupclosure{x}\), where \(M\)~is maximal and \(x\ne1\). Now let \(H\)~be the subgroup of~\(G\) consisting of \(p\)-th powers, and \(K\)~the subgroup consisting of elements of order~\(p\). Note \(H\)~and~\(K\) are just the range and kernel, respectively, of the map \(x\mapsto x^p\), and \(G/H\)~and~\(K\) are both elementary abelian. If possible, pull back a factorization of~\(G/H\) into~\(G\) and appeal to induction on a factor. Otherwise, appeal to the induction hypothesis on \(G/K\iso H\), and then pull back \(p\)-th roots of the generators from the factorization of~\(G/K\). For uniqueness, also use induction.
\end{proof}
\begin{app}
Classifying finite abelian groups.
\end{app}

\begin{thm}[Commutators]
Let \(G\)~be a group.
\begin{enumerate}[itemsep=0pt]
\item[(a)] For all \(x,y\in G\), \(xy=yx[x,y]\). In particular \(xy=yx\) iff \([x,y]=1\).
\item[(b)] For all \(H\subgroupeq G\), \(H\normsubgroupeq G\) iff \([H,G]\subgroupeq H\).
\item[(c)] \(G'\charsubgroupeq G\).
\item[(d)] For all \(N\normsubgroupeq G\), \(G/N\)~is abelian iff \(G'\subgroupeq N\). In particular, \(G/G'\)~is the largest abelian quotient of~\(G\).
\end{enumerate}
\end{thm}
\begin{app}
Determining when elements commute.
\end{app}

\begin{thm}[Internal direct products]
Let \(G\)~be a group. If \(H,K\normsubgroupeq G\) and \(H\sect K=1\), then \(HK\iso H\times K\).
\end{thm}
\begin{proof}[Proof idea]
Each element in~\(HK\) can be written uniquely in the form~\(hk\) for some \(h\in H\) and \(k\in K\). Also, since both \(H\)~and~\(K\) are normal, \([h,k]\in H\sect K=1\) for all \(h\in H\) and \(k\in K\), hence every element of~\(H\) commutes with every element of~\(K\). It follows that the map \(hk\mapsto (h,k)\) is the desired isomorphism.
\end{proof}
\begin{app}
Recognizing direct products.
\end{app}

\begin{thm}[Internal semidirect products]
Let \(G\)~be a group. If \(H,K\subgroupeq G\), \(H\normsubgroupeq G\), and \(H\sect K=1\), then \(HK\iso H\rtimes K\), where \(K\)~acts on~\(H\) by conjugation.
\end{thm}
\begin{proof}[Proof idea]
Again, the map \(hk\mapsto (h,k)\) is the desired isomorphism (but for different reasons!).
\end{proof}
\begin{app}
Recognizing semidirect products.
\end{app}

\subsection*{Techniques}
\begin{itemize}[itemsep=0pt]
\item Classifying finite abelian groups of order~\(n\) using the fundamental theorem:
\begin{enumerate}[itemsep=0pt]
\item Factor~\(n\) into its prime factorization \(n=p_1^{a_1}\cdots p_k^{a_k}\).
\item For each~\(i\), determine all possible abelian groups of order~\(p_i^{a_i}\) by first determining all partitions of the number~\(a_i\) of form \(a_i=b_1+\cdots+b_j\) with \(b_1\ge\cdots\ge b_j\ge1\), then forming corresponding direct products
\[Z_{p_i^{b_1}}\times\cdots\times Z_{p_i^{b_j}}\]
\item Determine all possible abelian groups of order~\(n\) by forming all direct products of the form \(A_1\times\cdots\times A_k\) where \(A_i\)~is a group from the previous step with \(\ord{A_i}=p_i^{a_i}\).
\end{enumerate}
\item Classifying finite groups of order~\(n\) using semidirect products:
\begin{enumerate}[itemsep=0pt]
\item Show that every group~\(G\) of order~\(n\) has proper subgroups \(H\)~and~\(K\) which form an internal semidirect product \(G=HK\) (for example, using Sylow's theorems).
\item Determine all possible isomorphism types for \(H\)~and~\(K\) (inductively).
\item For each pair \(H,K\) in the previous step, find all possible automorphism representations~\(\varphi\) of~\(K\) on~\(H\).
\item For each triple \(H,K,\varphi\) in the previous step, form the semidirect product \(H\rtimes_{\varphi}K\). 
\item Determine the distinct isomorphism types in the previous step.
\end{enumerate}
\item In inductive arguments involving finite abelian (or more generally, nilpotent) groups, taking quotients by maximal subgroups and leveraging the simple structure of the quotients.
\item Taking quotients to impose relations.
\item Proving normality using commutator subgroups.
\item Using direct and semidirect products to construct examples.
\end{itemize}
