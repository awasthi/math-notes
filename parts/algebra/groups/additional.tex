%
% Notes on Mathematics
% John Peloquin
%
% Algebra
% Groups
% Nilpotence, Solvability, Simplicity, and Freeness
%
\section{Nilpotence, Solvability, Simplicity, and Freeness}
\subsection*{Theorems}
\begin{thm}[Finite nilpotence]
If \(G\)~is a finite group, the following are equivalent:
\begin{enumerate}[itemsep=0pt]
\item[(a)] \(G\)~is nilpotent
\item[(b)] Normalizers grow in~\(G\) (that is, if \(H<G\), then \(H<N_G(H)\))
\item[(c)] Every Sylow subgroup of~\(G\) is normal
\item[(d)] \(G\)~is a direct product of Sylow subgroups
\item[(e)] Every maximal subgroup of~\(G\) is normal
\end{enumerate}
\end{thm}
\begin{proof}[Proof idea]
For (a)\(\implies\)(b), by induction (on the nilpotence class of~\(G\)). If \(Z(G)\subgroupeq H\), then \(\res{H}<\res{G}=G/Z(G)\), so by induction \(\res{H}<N_{\res{G}}(\res{H})=\res{N_G(H)}\), and hence by the lattice theorem \(H<N_G(H)\). If \(Z(G)\not\subgroupeq H\), then \(H<\groupclosure{H,Z(G)}\subgroupeq N_G(H)\).

For (b)\(\implies\)(c), by taking normalizers. If \(P\)~is a nonnormal Sylow subgroup of~\(G\), then \(P\normsubgroupeq N_G(P)<G\). But then \(P\)~is characteristic in~\(N_G(P)\), so \(P\normsubgroupeq N_G(N_G(P))\), so \(N_G(P)=N_G(N_G(P))\), contradicting that normalizers grow.

For (c)\(\implies\)(d), by internal direct product recognition.

For (d)\(\implies\)(a), by induction, since \(G/Z(G)\)~is also a product of Sylow subgroups.

Note (b)\(\implies\)(e) is immediate. 

For (e)\(\implies\)(c), use Frattini's argument. If \(P\)~is a nonnormal Sylow subgroup of~\(G\), then \(P\normsubgroupeq N_G(P)<M\normsubgroup G\) for some maximal subgroup~\(M\) of~\(G\). But by Frattini's argument, \(G=MN_G(P)=M\), a contradiction.
\end{proof}
\begin{app}
Finding large normalizers (for example, when proving non-simplicity), classifying nilpotent groups, taking quotients by maximal subgroups, etc.
\end{app}

\begin{cor}[\(p\)-groups]
\(p\)-groups are nilpotent.
\end{cor}
\begin{app}
Applying techniques for nilpotent groups to \(p\)-groups, and arbitrary finite groups via Sylow \(p\)-subgroups.
\end{app}

\begin{thm}[Universal property of free groups]
If \(S\)~is a set, \(G\)~is a group, and \(f:S\to G\) is any set map, there exists a unique homomorphism \(\varphi:F(S)\to G\) extending~\(f\).

Moreover, this property characterizes~\(F(S)\) up to a unique isomorphism which is the identity on~\(S\).
\end{thm}
\begin{proof}[Proof idea]
To satisfy the desired properties, \(\varphi\)~must be defined by
\[\varphi(\prod s_i^{\alpha_i})=\prod f(s_i)^{\alpha_i}\qquad(s_i\in S,\alpha_i\in\{-1,0,1\})\]
And since \(F(S)\)~is free of relations in~\(S\), \(\varphi\)~is well defined. This establishes existence and uniqueness of~\(\varphi\).

If \(F(S)\)~and~\(F^*(S)\) both satisfy the universal property, then there exist two unique homomorphisms \(\varphi:F(S)\to F^*(S)\) and \(\varphi^*:F^*(S)\to F(S)\) extending the identity map on~\(S\). Then by the universal property again, \(\varphi\varphi^*\)~and~\(\varphi^*\varphi\) are the identity, so \(\varphi\)~and~\(\varphi^*\) are inverses and \(F(S)\iso F^*(S)\).
\end{proof}

\subsection*{Techniques}
\begin{itemize}[itemsep=0pt]
\item Factoring nilpotent groups into parts using upper [lower] central series, and using induction on nilpotence class.
\item Factoring solvable groups into parts using commutator series, and using induction on solvable length.
\item In inductive arguments involving finite nilpotent groups, taking quotients by maximal subgroups and leveraging the simple structure of the quotients.
\item Proving a finite group is not simple:
\begin{itemize}[itemsep=0pt]
\item Counting elements using Sylow's theorem and showing that some proper Sylow subgroup must be normal.
\item Permutation representations on cosets. In particular, exploiting a lower bound on possible indices for proper subgroups of the simple group, working directly inside \(S_n\)~and~\(A_n\), etc.
\item Finding large normalizers of Sylow subgroups by working with multiple primes.
\item Finding large normalizers of intersections of Sylow subgroups.
\end{itemize}
\item Defining group actions on algebraic (and induced geometric) structures to prove existence and uniqueness of simple groups.
\item Taking quotients to impose relations.
\item Defining homomorphisms using free groups and presentations.
\end{itemize}
