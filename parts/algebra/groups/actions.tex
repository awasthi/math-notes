%
% Notes on Mathematics
% John Peloquin
%
% Algebra
% Groups
% Group Actions
%
\section{Group Actions}
\subsection*{Theorems}
\begin{thm}[Orbit partitioning]
Let \(G\)~act on~\(A\). The orbits of~\(A\) under~\(G\) partition~\(A\), so if \(a_1,\ldots,a_k\) are representatives from the orbits,
\[\ord{A}=\ord{O_{a_1}}+\cdots+\ord{O_{a_k}}\]
\end{thm}
\begin{proof}[Proof idea]
Orbits are equivalence classes.
\end{proof}
\begin{app}
Getting combinatorial information from group actions.
\end{app}

\begin{thm}[Orbit-stabilizer]
Let \(G\)~act on~\(A\). If \(a\in A\), \(\ord{O_a}=\idx{G_a}{G}\), so
\[\ord{G}=\ord{G_a}\ord{O_a}\]
\end{thm}
\begin{proof}[Proof idea]
By the map \(gG_a\mapsto g\cdot a\) (\(g\in G\)) and Lagrange.
\end{proof}
\begin{app}
Getting combinatorial information from group actions.
\end{app}
\begin{cor}[Cycle decomposition in~\(S_n\)]
If \(\sigma\in S_n\), \(\sigma\)~has a decomposition of the form
\[\sigma=(a_1\cdots a_{m_1})\cdots(a_{m_{k-1}+1}\cdots a_{m_k})\]
where the cycles partition \(\{1,\ldots,n\}\). This decomposition is unique up to the order of the cycles and cyclic permutation of the numbers in the cycles.
\end{cor}
\begin{proof}[Proof idea]
Let \(\sigma\)~act on \(\{1,\ldots,n\}\). Then the orbits uniquely determine the cycles.
\end{proof}
\begin{app}
Cycle shape, conjugacy classes in~\(S_n\).
\end{app}

\begin{thm}[Cayley]
If \(G\)~is a group, \(G\)~is isomorphic to some permutation group. In particular, if \(\ord{G}=n\), \(G\)~is isomorphic to a subgroup of~\(S_n\).
\end{thm}
\begin{proof}[Proof idea]
Let \(G\)~act on itself by left multiplication.
\end{proof}
\begin{app}
Getting permutations from abstract groups, showing that the study of abstract groups is (in a certain sense) equivalent to the study of permutation groups, getting combinatorial information about finite groups.
\end{app}

\begin{thm}[Class equation]
If \(G\)~is finite and \(g_1,\ldots,g_k\in G\) are representatives from the noncentral conjugacy classes in~\(G\), then
\[\ord{G}=\ord{Z(G)}+\sum_{i=1}^k\idx{C_G(g_i)}{G}\]
\end{thm}
\begin{proof}[Proof idea]
Let \(G\)~act on itself by conjugation, then apply the orbit partitioning and orbit-stabilizer theorems, noting that central conugacy classes are singletons.
\end{proof}
\begin{app}
Getting combinatorial information about groups, normal subgroups (which are unions of conjugacy classes), centers, centralizers, etc. and ruling out possibilities.
\end{app}

\begin{thm}[Sylow]
Let \(G\)~be a group with \(\ord{G}=p^a m\) where \(p\)~is prime not dividing~\(m\).
\begin{enumerate}[itemsep=0pt]
\item[(a)] There exists a Sylow \(p\)-subgroup of~\(G\).
\item[(b)] If \(P\)~is a Sylow \(p\)-subgroup of~\(G\) and \(Q\)~is a \(p\)-subgroup of~\(G\), there exists \(g\in G\) such that \(Q\subgroupeq gPg^{-1}\). In particular, all Sylow \(p\)-subgroups of~\(G\) are conjugate.
\item[(c)] If \(n_p\)~is the number of Sylow \(p\)-subgroups of~\(G\), \(n_p\equiv 1\) (mod~\(p\)). If \(P\)~is a Sylow \(p\)-subgroup of~\(G\), \(n_p=\idx{N_G(P)}{G}\), so \(n_p\)~divides~\(m\).
\end{enumerate}
\end{thm}
\begin{proof}[Proof idea]
For~(a), by induction on~\(\ord{G}\). If \(p\)~divides~\(\ord{Z(G)}\), then (by induction on abelian groups) there exists \(N\subgroupeq Z(G)\) with \(\ord{N}=p\). Now \(N\normsubgroupeq G\), so for \(\res{G}=G/N\), \(\ord{\res{G}}=p^{a-1}m\). By induction, \(\res{G}\)~has a subgroup \(\res{P}=P/N\) where \(N\subgroupeq P\subgroupeq G\) and \(\ord{\res{P}}=p^{a-1}\). Then \(\ord{P}=p^a\), so \(P\)~is a Sylow \(p\)-subgroup of~\(G\). If \(p\)~does not divide~\(\ord{Z(G)}\), by the class equation there must exist \(g\in G-Z(G)\) such that \(p^a\)~divides~\(\ord{C_G(g)}\). Since \(\ord{C_G(g)}<\ord{G}\), by induction \(C_G(g)\)~has a Sylow \(p\)-subgroup which is also a Sylow \(p\)-subgroup of~\(G\).

For (b)~and~(c), by group action. Fix a Sylow \(p\)-subgroup~\(P_1\) of~\(G\) and let \(\mathcal{C}=\{P_1,\ldots,P_k\}\) be the set of all \(G\)-conjugates of~\(P_1\). Then \(Q\)~acts on~\(\mathcal{C}\) by conjugation and partitions~\(\mathcal{C}\) into orbits of size \(\idx{N_G(P_i)\sect Q}{Q}=\idx{P_i\sect Q}{Q}\) (for representative~\(P_i\)). Taking \(Q=P_1\), it follows \(k\equiv1\) (mod~\(p\)). Moreover, there cannot be any \(p\)-subgroup~\(Q\) not contained in some~\(P_i\) lest \(k\not\equiv1\) (mod~\(p\)), so (b)~holds. Now \(\mathcal{C}\)~just consists of the Sylow \(p\)-subgroups of~\(G\), so \(n_p=k\) and (c)~follows.
\end{proof}
\begin{app}
Breaking groups into simpler pieces (\(p\)-subgroups), finding normal subgroups, proving groups are not simple, classifying groups.
\end{app}

\subsection*{Techniques}
\begin{itemize}[itemsep=0pt]
\item Getting information about groups using group actions:
\begin{itemize}[itemsep=0pt]
\item On arbitrary sets.
\item By left multiplication on cosets and elements of the group.
\item By conjugation on subsets of the group.
\end{itemize}
\item Applying combinatorial results (orbit partitioning, orbit-stabilizer, the class equation).
\item Getting information about (quotients of) normalizers and centralizers from information about automorphism groups.
\item Using induction on the order of a group.
\item Restricting the source of group actions to subgroups about which we know more, to get additional combinatorial information.
\item Breaking down groups with Sylow's theorems.
\item Using permutation representations to prove groups isomorphic to subgroups of \(S_n\)~and~\(A_n\).
\item Using cycle shapes to determine conjugacy classes in~\(S_n\).
\end{itemize}
