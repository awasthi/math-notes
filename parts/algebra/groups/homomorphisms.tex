%
% Notes on Mathematics
% John Peloquin
%
% Algebra
% Groups
% Homomorphisms and Quotients
%
\section{Homomorphisms and Quotients}
\subsection*{Theorems}
\begin{thm}[Normality]
If \(G\)~is a group and \(N\subgroupeq G\), the following are equivalent:
\begin{enumerate}[itemsep=0pt]
\item[(a)] \(N\normsubgroupeq G\)
\item[(b)] \(gNg^{-1}\subseteq N\) for all \(g\in G\)
\item[(c)] \(gN=Ng\) for all \(g\in G\)
\item[(d)] \(N_G(N)=G\)
\item[(e)] the coset space \(G/N\) forms a group under the usual operation
\item[(f)] \(N\)~is the kernel of a homomorphism on~\(G\)
\end{enumerate}
\end{thm}

\begin{thm}[Tower law]
If \(G\)~is a group and \(H\subgroupeq K\subgroupeq G\), then
\[\idx{H}{G}=\idx{K}{G}\idx{H}{K}\]
\end{thm}
\begin{proof}[Proof idea]
The cosets of~\(K\) partition~\(G\), and the number of \(G\)-cosets of~\(H\) in any coset~\(gK\) of~\(K\) is equal to~\(\idx{H}{K}\), by the map \(kH\mapsto gkH\) (\(k\in K\)).
\end{proof}

\begin{cor}[Lagrange]
If \(G\)~is finite, \(\ord{K}\)~divides~\(\ord{G}\) and the number of left [right] cosets of~\(K\) in~\(G\) is \(\ord{G}/\ord{K}\).
\end{cor}
\begin{app}
Getting combinatorial information about subgroups and ruling out possibilities.
\end{app}

\begin{thm}[Isomorphism theorems]
Let \(G\)~be a group. 
\begin{enumerate}[itemsep=0pt]
\item[(a)] If \(\varphi:G\to H\) is a surjective homomorphism and \(K=\ker\varphi\), then \(G/K\iso H\).
\item[(b)] If \(H,K\subgroupeq G\) and \(H\subgroupeq N_G(K)\), then \(HK/K\iso H/H\sect K\).
\item[(c)] If \(H,K\normsubgroupeq G\) and \(H\subgroupeq K\), then \((G/H)/(K/H)\iso G/K\).
\end{enumerate}
\end{thm}
\begin{proof}[Proof idea]
For~(a), by the isomorphism \(gK\mapsto\varphi(g)\).

For~(b), by~(a). Define \(\varphi:H\to HK/K\) by \(h\mapsto hK\). Then \(\varphi\)~is surjective and \(\ker\varphi=H\sect K\), so \(H/H\sect K\iso HK/K\).

For~(c), by~(a). Define \(\varphi:G/H\to G/K\) by \(gH\mapsto gK\). Then \(\varphi\)~is surjective and \(\ker\varphi=K/H\), so \((G/H)/(K/H)\iso G/K\).
\end{proof}

\begin{thm}[Lattice theorem]
If \(G\)~is a group and \(N\normsubgroupeq G\), there is a bijection between the subgroups~\(H\) of~\(G\) containing~\(N\) and the subgroups \(\res{H}=H/N\) of \(\res{G}=G/N\) which respects subgroup lattice structure. More specifically, for all \(H,K\subgroupeq G\) with \(N\subgroupeq H,K\),
\begin{enumerate}[itemsep=0pt]
\item[(a)] \(H\subgroupeq K\) iff \(\res{H}\subgroupeq\res{K}\).
\item[(b)] \(H\normsubgroupeq K\) iff \(\res{H}\normsubgroupeq\res{K}\).
\item[(c)] \(\res{\groupclosure{H,K}}=\groupclosure{\res{H},\res{K}}\).
\item[(d)] \(\res{H\sect K}=\res{H}\sect\res{K}\).
\item[(e)] if \(H\subgroupeq K\), \(\idx{H}{K}=\idx{\res{H}}{\res{K}}\).
\end{enumerate}
\end{thm}
\begin{app}
Relating the structures of \(G\)~and~\(G/N\), often in inductive arguments; showing that the lattice of~\(G/N\) is the part of the lattice of~\(G\) above~\(N\).
\end{app}

\subsection*{Techniques}
\begin{itemize}[itemsep=0pt]
\item Determining subgroup lattice structure with the tower law (Lagrange).
\item Proving normality:
\begin{itemize}[itemsep=0pt]
\item Directly (conjugation).
\item Showing left cosets equal right cosets.
\item Determining normalizers (with help of the tower law, etc).
\item Exhibiting kernels.
\item Special cases (for example, subgroups of the center).
\end{itemize}
\item Using induction on the order of a group:
\begin{itemize}[itemsep=0pt]
\item Involving subgroups.
\item Involving quotients, often by subgroups of the center (since these are always normal).
\end{itemize}
\item Factoring groups into parts using composition series (Jordan-H\"older).
\item Classifying finite groups using the Jordan-H\"older program:
\begin{itemize}[itemsep=0pt]
\item Classifying all finite simple groups.
\item Finding all ways to assemble simple groups into larger groups.
\end{itemize}
\end{itemize}
