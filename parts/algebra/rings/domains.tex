%
% Notes on Mathematics
% John Peloquin
%
% Algebra
% Rings
% Domains
%
\section[EDs, PIDs, UFDs, and IDs]{Euclidean Domains, Principal Ideal Domains, Unique Factorization Domains, and Integral Domains}
\subsection*{Theorems}
\begin{thm}[Ideals and greatest common divisors]
If \(R\)~is a ring and \(a,b\in R\), then \(d\in R\)~is a greatest common divisor of \(a\)~and~\(b\) iff \((d)\)~is the smallest principal ideal containing~\((a,b)\). 
\end{thm}
\begin{app}
Translating between arithmetic properties and properties of ideals.
\end{app}
\begin{cor}
Greatest common divisors are unique up to units (i.e., they are associates).
\end{cor}

\begin{thm}[Euclidean algorithm]
If \(R\)~is a euclidean domain and \(a,b\in R\) are nonzero with division \(a=bq+r\) (where \(q,r\in R\) with \(r=0\) or \(N(r)<N(b)\)), then
\[(a,b)=(b,r)\]
In particular, if \(r_n\)~is the last nonzero remainder in the euclidean algorithm with \(a\)~and~\(b\) (where \(r_0=b\)), then \((a,b)=(r_n)\), so
\begin{enumerate}[itemsep=0pt]
\item[(i)] \(r_n\)~is a greatest common divisor of \(a\)~and~\(b\), and
\item[(ii)] \(r_n=ax+by\) for some \(x,y\in R\).
\end{enumerate}
\end{thm}
\begin{proof}[Proof idea]
By induction,
\[(a,b)=(b,r_1)=(r_1,r_2)=\cdots=(r_n,r_{n+1})=(r_n)\]
where the last equality holds since \(r_{n+1}=0\). Now (i)~and~(ii) are immediate.
\end{proof}
\begin{app}
Computing greatest common divisors (efficiently), computing multiplicative inverses of coprime elements, etc.
\end{app}

\begin{thm}[EDs are PIDs]
Euclidean domains are principal ideal domains.
\end{thm}
\begin{proof}[Proof idea]
By division with remainder.

If \(R\)~is a euclidean domain and \(I\subseteq R\) is a nonzero ideal, fix \(b\in I\) nonzero with minimum norm~\(N(b)\). Then for any \(a\in I\), there exist \(q,r\in R\) with \(a=bq+r\) where \(r=0\) or \(N(r)<N(b)\). Now \(r\in I\), so by minimality of~\(N(b)\), \(r=0\) and \(a=bq\in(b)\). Thus \(I=(b)\) is principal.
\end{proof}

\begin{thm}[Prime and maximal ideals]
\ 
\begin{enumerate}[itemsep=0pt]
\item[(a)] In an integral domain, maximal ideals are prime.
\item[(b)] In a principal ideal domain, an ideal is prime iff it is maximal. 
\end{enumerate}
\end{thm}
\begin{proof}[Proof idea]
For~(a), by previous results.

For~(b), by direct computation.
\end{proof}

\begin{thm}[Primes and irreducibles]
\ 
\begin{enumerate}[itemsep=0pt]
\item[(a)] In an integral domain, primes are irreducible.
\item[(b)] In a principal ideal domain, an element is prime iff it is irreducible.
\item[(c)] In a unique factorization domain, an element is prime iff it is irreducible.
\end{enumerate}
\end{thm}
\begin{proof}[Proof idea]
For~(a), by direct computation.

For~(b), if \(p\)~is irreducible, argue that \((p)\)~is maximal.

For~(c), if \(p\)~is irreducible and \(p|ab\), factor both sides and argue that \(p\)~must divide one of the irreducible factors in \(a\)~or~\(b\).
\end{proof}

\begin{thm}[PIDs are UFDs]
Principal ideal domans are unique factorization domains.
\end{thm}
\begin{proof}[Proof idea]
For existence of factorizations, by a subdivision argument.

If \(R\)~is a principal ideal domain and \(a\in R\) cannot be factored into finitely many irreducibles, then a factorization tree for~\(a\) can be arbitrarily extended in height. By recursion and the axiom of choice, there exists an infinite descending sequence of proper divisors
\[\cdots\ |\ a_k\ |\ \cdots\ |\ a_2\ |\ a_1\ |\ a\]
This means there is an infinite properly ascending chain of ideals
\[(a)\subset(a_1)\subset(a_2)\subset\cdots\subset(a_k)\subset\cdots\]
But this chain must eventually collapse since \(I=\bigunion_{i=1}^{\infty}(a_i)\) is a principal ideal of form \(I=(b)\), so \(b\in(a_i)\) for some~\(i\) and
\[I=(b)\subseteq(a_i)\subseteq(a_{i+1})\subseteq\cdots\subseteq I\]
---a contradiction. Therefore factorizations exist.

For uniqueness of factorizations, use the primality of irreducibles to argue that the irreducibles in any two factorizations of an element must be associates.
\end{proof}

\begin{thm}[GCDs in UFDs]
If \(R\)~is a unique factorization domain and \(a,b\in R\) with
\[a=u\prod_{i=1}^k p_i^{e_i}\qquad b=v\prod_{i=1}^k p_i^{f_i}\]
where \(u,v\in R\) are units, \(p_1,\ldots,p_k\in R\) are primes, and \(e_i,f_j\ge0\), then
\[d=\prod_{i=1}^k p_i^{\min(e_i,f_i)}\]
is a greatest common divisor of \(a\)~and~\(b\).
\end{thm}
\begin{proof}[Proof idea]
By unique factorization.
\end{proof}
\begin{app}
Computing greatest common divisors from factorizations.
\end{app}

\begin{thm}[Fundamental theorem of arithmetic]
\(\Z\)~is a euclidean domain, hence a princpal ideal domain and unique factorization domain. In particular, every integer can be factored uniquely as a product of primes.
\end{thm}
\begin{proof}[Proof idea]
By induction.
\end{proof}
\begin{app}
Arithmetic with the integers.
\end{app}

\begin{rmk}
By definitions and results,
\begin{multline*}
\text{fields}\subseteq\text{euclidean domains (EDs)}\\
\subseteq\text{principal ideal domains (PIDs)}\subseteq\text{unique factorization domains (UFDs)}\\
\subseteq\text{integral domains (IDs)}\subseteq\text{commutative rings}
\end{multline*}
Each of these inclusions is proper.
\end{rmk}

\subsection*{Techniques}
\begin{itemize}[itemsep=0pt]
\item Translating between arithmetic and ideal properties.
\item Computing with the euclidean algorithm.
\item Computing with principal ideals.
\item Factoring complex elements into simpler pieces.
\item Subdivision to construct ascending chains of ideals and take limits.
\end{itemize}
