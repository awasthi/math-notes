%
% Notes on Mathematics
% John Peloquin
%
% Algebra
% Rings
% Polynomials
\section{Polynomials}
\subsection*{Theorems}
Let \(R\)~be a commutative ring with identity \(1\ne0\).
\begin{thm}[Reduction of coefficients]
If \(I\)~is an ideal in~\(R\), then
\[R[x]/I[x]\iso(R/I)[x]\]
\end{thm}
\begin{proof}[Proof idea]
By the first isomorphism theorem.
\end{proof}
\begin{app}
Diophantine equations, irreducibility criteria (e.g. Eisenstein).
\end{app}

\begin{thm}[Polynomial rings over integral domains]
\(R\)~is an integral domain iff \(R[x]\)~is an integral domain, in which case
\[\deg p(x)q(x)=\deg p(x)+\deg q(x)\]
for all \(p(x),q(x)\in R[x]\).
\end{thm}
\begin{proof}[Proof idea]
By looking at leading terms.

If \(R\)~is an integral domain and \(p(x),q(x)\in R[x]\) are nonzero with leading terms \(a_nx^n\)~and~\(b_m x^m\), respectively, then since \(a_n\)~and~\(b_m\) are nonzero, \(a_nb_m\)~is nonzero and the leading term of~\(p(x)q(x)\) is \(a_nb_m x^{n+m}\).
\end{proof}
\begin{app}
Computation of degree.
\end{app}
\begin{cor}
The units in \(R[x]\)~are just the units in~\(R\).
\end{cor}

\begin{thm}[Polynomial rings over fields]
\(R\)~is a field iff \(R[x]\)~is a euclidean domain with norm given by degree, in which case for all \(a(x),b(x)\in R[x]\) with \(b(x)\ne0\), there exist unique \(q(x),r(x)\in R[x]\) with
\[a(x)=b(x)q(x)+r(x)\qquad\text{where }\deg r(x)<\deg b(x)\]
\end{thm}
\begin{proof}[Proof idea]
For the forward direction, by induction on degree.

Let \(a(x)=a_nx^n+\cdots+a_0\) and \(b(x)=b_mx^m+\cdots+b_0\). If \(n<m\), take \(q(x)=0\) and \(r(x)=a(x)\). Otherwise note \(a(x)-(a_n/b_m)x^{n-m}b(x)\) has degree less than~\(n\), so by induction there exist \(s(x),r(x)\in R[x]\) such that
\[a(x)-\frac{a_n}{b_m}x^{n-m}b(x)=b(x)s(x)+r(x)\qquad\text{where }\deg r(x)<\deg b(x)\]
Take \(q(x)=s(x)+(a_n/b_m)x^{n-m}\). To prove uniqueness, compute degrees.

For the reverse direction, note if \(R[x]\)~is a principal ideal domain then \(R\iso R[x]/(x)\) is a field since \((x)\)~is prime and hence maximal in~\(R[x]\).
\end{proof}
\begin{app}
Polynomial division with remainder, equivalence of roots and linear factors, irreducibility criteria, field extensions, etc.
\end{app}
\begin{cor}
If \(F\)~is a field and \(f(x)\in F[x]\), \(F[x]/(f(x))\)~is a field iff \(f(x)\)~is irreducible in~\(F[x]\).
\end{cor}
\begin{proof}[Proof idea]
Because \(F[x]\)~is a principal ideal domain, \(f(x)\)~is irreducible iff \(f(x)\)~is prime iff \((f(x))\)~is maximal.
\end{proof}
\begin{app}
Constructing simple algebraic field extensions.
\end{app}
\begin{cor}
If \(F\)~is a field and \(f(x)\in F[x]\) has unique factorization
\[f(x)=f_1(x)^{n_1}\cdots f_k(x)^{n_k}\]
where the~\(f_i(x)\) are pairwise distinct irreducibles, then
\[F[x]/(f(x))\iso\prod_{i=1}^k F[x]/(f_i(x)^{n_i})\]
\end{cor}
\begin{proof}[Proof idea]
By the chinese remainder theorem.
\end{proof}

\begin{thm}[Gauss]
If \(R\)~is a unique factorization domain with fraction field~\(F\) and \(p(x)\in R[x]\) is reducible in~\(F[x]\), then \(p(x)\)~is reducible in~\(R[x]\). More specifically, if \(p(x)=a(x)b(x)\) with \(a(x),b(x)\in F[x]\), there exist \(r,s\in F\) such that \(ra(x),sb(x)\in R[x]\) and \(p(x)=ra(x)sb(x)\).

Conversely, if the greatest common divisor of the coefficients of~\(p(x)\) is~\(1\) and \(p(x)\)~is reducible in~\(R[x]\), then \(p(x)\)~is reducible in~\(F[x]\).
\end{thm}
\begin{proof}[Proof idea]
For the forward direction, by clearing denominators in the factorization and then cancelling out each of the irreducible factors of the common denominator.

Let \(d\)~be a common denominator of all of the coefficients of \(a(x)\)~and~\(b(x)\), so
\[dp(x)=a'(x)b'(x)\]
where \(a'(x)\in R[x]\) and \(b'(x)\in R[x]\) are \(R\)-multiples of \(a(x)\)~and~\(b(x)\), respectively. If \(d\)~is not a unit, write \(d=p_1\cdots p_k\) where each~\(p_i\) is irreducible, and hence prime, in~\(R\). For each~\(i\), \(\res{a'(x)b'(x)}=\res{0}\) in the integral domain \((R/(p_i))[x]\), so \(p_i\)~must divide one of \(a'(x)\)~or~\(b'(x)\) in~\(R[x]\). Therefore each~\(p_i\) in the above equation can be cancelled out while keeping the equation in~\(R[x]\). It follows that \(p(x)\)~reduces in~\(R[x]\) into \(F\)-multiples of \(a(x)\)~and~\(b(x)\), as desired.

The reverse direction is trivial.
\end{proof}
\begin{app}
Relating reducibility in \(R[x]\)~and~\(F[x]\), unique factorization in~\(R[x]\), transferring irreducibility criteria from~\(F[x]\) to~\(R[x]\), etc.
\end{app}

\begin{cor}[Polynomial rings over unique factorization domains]
\(R\)~is a unique factorization domain iff \(R[x]\)~is a unique factorization domain.
\end{cor}
\begin{proof}[Proof idea]
By Gauss, pulling back unique factorization from~\(F[x]\) to~\(R[x]\).
\end{proof}

\begin{thm}[Linear factors and roots]
If \(F\)~is a field, \(f(x)\in F[x]\) has a linear factor \((x-\alpha)\) iff \(f(\alpha)=0\).
\end{thm}
\begin{proof}[Proof idea]
For the forward direction by substitution, and for the reverse direction by division with remainder.
\end{proof}
\begin{cor}
\(f(x)\)~has at most \(\deg f(x)\)~roots, even counting multiplicity.
\end{cor}
\begin{cor}
If \(\deg f(x)\)~is \(2\)~or~\(3\), \(f(x)\)~is irreducible in~\(F[x]\) iff it has no roots in~\(F\).
\end{cor}
\begin{app}
Determining irreducibility of polynomials of low degree.
\end{app}

\begin{thm}[Rational roots]
If \(R\)~is a unique factorization domain with fraction field~\(F\) and \(p(x)=a_nx^n+\cdots+a_0\in R[x]\) has a root \(r/s\in F\) where \(r,s\in R\) and \((r,s)=1\), then \(s|a_n\) and \(r|a_0\) in~\(R\).
\end{thm}
\begin{proof}[Proof idea]
By direct computation of~\(p(r/s)\) and clearing denominators.
\end{proof}
\begin{app}
Finding rational roots, determining irreducibility of polynomials of low degree.
\end{app}

\begin{thm}[Reduction and irreducibility]
If \(R\)~is an integral domain, \(f(x)\in R[x]\) is nonconstant monic, and \(I\)~is a proper ideal in~\(R\) such that \(\res{f(x)}\)~cannot be properly factored in~\((R/I)[x]\), then \(f(x)\)~is irreducible in~\(R[x]\).
\end{thm}
\begin{app}
Determining irreducibility of polynomials.
\end{app}

\begin{cor}[Eisenstein]
If \(R\)~is an integral domain with fraction field~\(F\), \(p(x)=x^n+a_{n-1}x^{n-1}+\cdots+a_0\in R[x]\), and \(P\)~is a prime ideal in~\(R\) such that \(a_{n-1},\ldots,a_0\in P\) and \(a_0\not\in P^2\), then \(p(x)\)~is irreducible in \(R[x]\)~and~\(F[x]\).
\end{cor}
\begin{proof}[Proof idea]
By reducing coefficients mod~\(P\).

If \(p(x)=a(x)b(x)\) in~\(R[x]\), then \(\res{a(x)b(x)}=\res{x}^n\) in the integral domain~\((R/P)[x]\). Therefore the constant terms of \(\res{a(x)}\)~and~\(\res{b(x)}\) must both be~\(\res{0}\), that is, the constant terms of \(a(x)\)~and~\(b(x)\) must both be in~\(P\), so \(a_0\in P^2\)---a contradiction. Irreducibility in~\(F[x]\) follows from Gauss.
\end{proof}
\begin{app}
Determining irreducibility of polynomials.
\end{app}

\subsection*{Techniques}
\begin{itemize}[itemsep=0pt]
\item Induction on degree of polynomials.
\item Looking at leading and constant terms of products of polynomials.
\item Division with remainder of polynomials.
\item Reducing coefficients of polynomials.
\item Transferring properties between \(R\)~and~\(R[x]\):
\begin{itemize}[itemsep=0pt]
\item \(R\)~commutative \(\iff\) \(R[x]\)~commutative
\item \(R\)~ID \(\iff\) \(R[x]\)~ID
\item \(R\)~UFD \(\iff\) \(R[x]\)~UFD
\item \(R\)~field \(\iff\) \(R[x]\)~PID \(\iff\) \(R[x]\)~ED
\end{itemize}
\item Transferring properties between \(R[x]\)~and~\(F[x]\).
\item Irreducibility criteria:
\begin{itemize}[itemsep=0pt]
\item Roots (for polynomials of low degree)
\item Reducing coefficients
\item Eisenstein
\item Substitution
\end{itemize}
\end{itemize}
