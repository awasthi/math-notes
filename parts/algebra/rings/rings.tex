%
% Notes on Mathematics
% John Peloquin
%
% Algebra
% Rings
% Rings and Ideals
%
\section{Rings and Ideals}
\subsection*{Theorems}
\begin{thm}[Ideals]
If \(R\)~is a ring and \(I\)~is an additive subgroup of~\(R\), the following are equivalent:
\begin{enumerate}[itemsep=0pt]
\item[(a)] \(I\)~is an ideal
\item[(b)] the additive coset space~\(R/I\) forms a ring under the natural operations
\item[(c)] \(I\)~is the kernel of a ring homomorphism on~\(R\)
\end{enumerate}
\end{thm}

\begin{thm}[Isomorphism theorems]
Let \(R\)~be a ring.
\begin{enumerate}[itemsep=0pt]
\item[(a)] If \(\varphi:R\to S\) is a surjective ring homomorphism and \(I=\ker\varphi\), then \(R/I\iso S\).
\item[(b)] If \(I\)~is a subring of~\(R\) and~\(J\) is an ideal of~\(R\), then \(I+J/J\iso I/I\sect J\).
\item[(c)] If \(I\)~and~\(J\) are ideals of~\(R\) with \(I\subseteq J\), then \((R/I)/(J/I)\iso R/J\).
\end{enumerate}
\end{thm}
\begin{proof}[Proof idea]
Apply the group-theoretic isomorphism theorems, then argue directly that the group isomorphisms are ring isomorphisms.
\end{proof}

\begin{thm}[Lattice theorem]
If \(R\)~is a ring and \(I\)~is an ideal in~\(R\), there is a bijective correspondence between the subrings~\(J\) of~\(R\) containing~\(I\) and the subrings \(\res{J}=J/I\) of \(\res{R}=R/I\) which respects subring lattice structure and preserves ideals.
\end{thm}
\begin{proof}[Proof idea]
Apply the group-theoretic lattice theorem, then argue directly that the correspondence preserves ideals.
\end{proof}
\begin{app}
Relating the structures of \(R\)~and~\(R/I\).
\end{app}

\begin{thm}[Cancellation in integral domains]
If \(R\)~is an integral domain, \(a,b,c\in R\), \(a\ne0\), and \(ab=ac\), then \(b=c\).
\end{thm}

\begin{thm}[Ideals in fields]
If \(R\)~is a commutative ring with identity \(1\ne0\), then \(R\)~is a field iff the only ideals in~\(R\) are \((0)\)~and~\((1)\).
\end{thm}
\begin{app}
Proving that rings are fields, or that ideals are trivial.
\end{app}

\begin{thm}[Quotients]
Let \(R\)~be a commutative ring with identity \(1\ne0\) and \(I\)~an ideal in~\(R\).
\begin{enumerate}[itemsep=0pt]
\item[(a)] \(R/I\)~is a field iff \(I\)~is maximal.
\item[(b)] \(R/I\)~is an integral domain iff \(I\)~is prime.
\end{enumerate}
\end{thm}
\begin{proof}[Proof idea]
By the lattice theorem, for~(a) using the ideal structure of fields and for~(b) using definitions.
\end{proof}
\begin{app}
Proving that quotients are fields [integral domains], or that ideals are maximal [prime], constructing field extensions.
\end{app}

\begin{cor}
Maximal ideals are prime.
\end{cor}
\begin{rmk}
The converse of the corollary is false in general, but it is true in principal ideal domains.
\end{rmk}

\begin{thm}[Fields of fractions]
If \(R\)~is an integral domain, there is a unique smallest field~\(F\) containing~\(R\), in the sense that
\begin{enumerate}[itemsep=0pt]
\item[(i)] \(F\)~contains a subring isomorphic to~\(R\), and
\item[(ii)] if \(K\)~is any field containing a subring isomorphic to~\(R\), then \(K\)~also contains an extension of that subring isomorphic to~\(F\). Moreover, the extension is just the subfield of~\(K\) generated by the subring.
\end{enumerate}
\end{thm}
\begin{proof}[Proof idea]
Construct the field of `fractions' over~\(R\) by taking equivalence classes of pairs over~\(R\), then defining addition and multiplication in the natural ways. Embed~\(R\) in the natural way. Argue that any field containing~\(R\) must contain all the `fractions', hence must contain~\(F\).
\end{proof}
\begin{app}
Enabling one to do computations in a field over an integral domain, which might be more convenient (for example, when working with integers, or with polynomials whose coefficients lie in an integral domain, etc.).
\end{app}

\begin{thm}[Chinese remainder theorem]
If \(R\)~is a commutative ring with identity \(1\ne0\) and \(A_1,\ldots,A_k\) are pairwise comaximal ideals in~\(R\), then the projection map
\[\varphi:R\to\prod_{i=1}^k R/A_i\]
is a surjective ring homomorphism with \(\ker\varphi=A_1\sect\cdots\sect A_k=A_1\cdots A_k\), so
\[R/A_1\cdots A_k\iso\prod_{i=1}^k R/A_i\]
\end{thm}
\begin{proof}[Proof idea]
By induction on \(k\ge 2\).

For \(k=2\), use the fact that there exist \(x\in A_1\) and \(y\in A_2\) with \(x+y=1\) to show that \(\varphi\)~is surjective and \(A_1\sect A_2=A_1A_2\), then appeal to the first isomorphism theorem.

For \(k>2\), argue that \(A_1\)~and~\(A_2\cdots A_k\) are comaximal, then appeal to \(k=2\).
\end{proof}

\begin{app}
Solving simultaneous congruences, etc.
\end{app}

\begin{cor}
The Euler \(\varphi\)~function is multiplicative, that is
\[\varphi(mn)=\varphi(m)\varphi(n)\qquad(m,n)=1\]
\end{cor}
\begin{proof}[Proof idea]
If \((m,n)=1\), then \(\Z/mn\Z\iso\Z/m\Z\times\Z/n\Z\), hence in particular
\[(\Z/mn\Z)^{\times}\iso(\Z/m\Z)^{\times}\times(\Z/n\Z)^{\times}\]
The order on the left is~\(\varphi(mn)\), and the order on the right is~\(\varphi(m)\varphi(n)\).
\end{proof}

\subsection*{Techniques}
\begin{itemize}[itemsep=0pt]
\item Using group-theoretic results to help establish ring-theoretic results.
\item Using induction on the order of finite rings, involving ideals and quotients.
\item Relating ideal structure to global structure.
\item Computing in fraction fields.
\item Using the chinese remainder theorem to solve simultaneous congruences, etc.
\end{itemize}
