%
% Notes on Mathematics
% John Peloquin
%
% Algebra
% Fields
% Fields and Field Extensions
%
\section{Fields and Field Extensions}
\subsection*{Theorems}
Let \(F\)~be a field.
\begin{thm}[Field homomorphisms]
If \(R\)~is a ring and \(\varphi:F\to R\) is a nonzero ring homomorphism, then \(\varphi\)~is injective.
\end{thm}
\begin{proof}[Proof idea]
By the ideal structure of fields, \(\ker\varphi=(0)\).
\end{proof}
\begin{app}
Showing that we cannot study fields by taking quotients (as with groups or rings), but by embedding them in larger rings.
\end{app}

\begin{thm}[Tower law]
If \(F\subseteq L\subseteq K\) is a tower of fields, then
\[[K:F]=[K:L][L:F]\]
In particular, \([L:F]\)~divides~\([K:F]\).
\end{thm}
\begin{proof}[Proof idea]
By direct computation with bases.
\end{proof}
\begin{app}
Structure of extensions.
\end{app}

\begin{thm}[Composites of finite extensions]
If \(K\)~and~\(L\) are finite over~\(F\) (with both contained in some common extension), then \(KL\)~is finite over~\(F\) and
\[[KL:F]\le[K:F][L:F]\]
If \([K:F]\)~and~\([L:F]\) are relatively prime, then equality holds.
\end{thm}
\begin{proof}[Proof idea]
By direct computation with bases and the tower law.
\end{proof}
\begin{app}
Structure of extensions.
\end{app}

\begin{thm}[Prime subfield]
\(F\)~has a subfield isomorphic to either~\(\F_p\) (if \(\charsubgroupeq F=p\)) or~\(\Q\) (if \(\charsubgroupeq F=0\)).
\end{thm}
\begin{proof}[Proof idea]
Take the subfield generated by~\(1\) in~\(F\).
\end{proof}
\begin{app}
Viewing all fields as extension fields.
\end{app}

\begin{thm}[Existence of simple algebraic extensions]
If \(f(x)\in F[x]\) is nonconstant, there exists a field~\(K\) extending~\(F\) and containing a root~\(\alpha\) of~\(f(x)\).
\end{thm}
\begin{proof}[Proof idea]
By construction, by adjoining a new element and taking a quotient to impose the root relation. 

Assume without loss of generality that \(f(x)\)~is irreducible over~\(F\) and set \(K=F[x]/(f(x))\). Then \(K\)~is a field containing an isomorphic copy of~\(F\), and if \(\alpha\)~denotes the image of~\(x\) in~\(K\), \(f(\alpha)=0\) by construction.
\end{proof}
\begin{app}
Adjoining a root of a polynomial, existence of splitting fields.
\end{app}

\begin{thm}[Structure of simple algebraic extensions]
If \(f(x)\in F[x]\) is irreducible over~\(F\) with \(n=\deg f(x)\), and \(K=F[x]/(f(x))\) with \(\alpha\)~the image of~\(x\) in~\(K\), then the elements \(1,\alpha,\ldots,\alpha^{n-1}\) form a basis for~\(K\) over~\(F\). In particular, \([K:F]=n\) and
\[K=\{\,a_0+a_1\alpha+\cdots+a_{n-1}\alpha^{n-1}\mid a_0,\ldots,a_{n-1}\in F\,\}\]
\end{thm}
\begin{proof}[Proof idea]
For spanning, by division by~\(f(x)\) with remainder in~\(F[x]\), and for linear independence, by irreducibility of~\(f(x)\) over~\(F\).
\end{proof}
\begin{app}
Computing in simple algebraic extensions, showing the finiteness of finitely generated algebraic extensions.
\end{app}

\begin{thm}[Structure of simple algebraic extensions]
If \(f(x)\in F[x]\) is irreducible over~\(F\) and \(K\)~is a field extending~\(F\) and containing a root~\(\alpha\) of~\(f(x)\), then \(F(\alpha)\iso F[x]/(f(x))\).
\end{thm}
\begin{proof}[Proof idea]
By mapping.

Let \(\varphi:F[x]\to F(\alpha)\) map \(g(x)\mapsto g(\alpha)\). Then \(\varphi\)~factors through~\((f(x))\) to the field mapping
\[\res{\varphi}:F[x]/(f(x))\to F(\alpha)\]
Now \(\res{\varphi}\)~is injective since it is nonzero, and it is surjective since its image is a subfield of~\(K\) containing \(F\)~and~\(\alpha\), and hence~\(F(\alpha)\).
\end{proof}

\begin{thm}[Uniqueness of simple algebraic extensions]
Let \(\varphi:F\to F^*\) be an isomorphism, \(f(x)\in F[x]\) be irreducible over~\(F\), and \(f^*(x)\in F^*[x]\) correspond to~\(f(x)\) under~\(\varphi\). If \(\alpha\)~is any root of~\(f(x)\) in an extension~\(K\) of~\(F\) and \(\beta\)~is any root of~\(f^*(x)\) in an extension~\(K^*\) of~\(F^*\), then \(\varphi\)~extends to an isomorphism \(\Phi:F(\alpha)\to F^*(\beta)\):
\begin{center}
\begin{tabular}{rlll}
\(\Phi:\)&\(F(\alpha)\)&\(\longrightarrow\)&\(F^*(\beta)\)\\
&\(\;\Bigl|\)&&\(\;\Bigr|\)\\
\(\varphi:\)&\(F\)&\(\longrightarrow\)&\(F^*\)
\end{tabular}
\end{center}
In particular, if \(\alpha,\beta\) are any two roots of~\(f(x)\), \(F(\alpha)\iso F(\beta)\).
\end{thm}
\begin{proof}[Proof idea]
By the structure of simple algebraic extensions.

Note \(\varphi\)~naturally extends to the ring isomorphism \(F[x]\to F^*[x]\) mapping \(f(x)\mapsto f^*(x)\), so \(\varphi\)~naturally extends to an isomorphism
\begin{equation*}
F(\alpha)\iso F[x]/(f(x))\iso F^*[x]/(f^*(x))\iso F^*(\beta)\qedhere
\end{equation*}
\end{proof}
\begin{app}
Algebraic indistinguishability of roots, uniqueness of splitting fields.
\end{app}

\begin{thm}[Minimal polynomial]
If \(\alpha\)~is algebraic over~\(F\), there exists a unique monic irreducible polynomial~\(m_{\alpha,F}(x)\in F[x]\) such that \(m_{\alpha,F}(\alpha)=0\). If \(f(x)\in F[x]\) and \(f(\alpha)=0\), then \(m_{\alpha,F}(x)\)~divides~\(f(x)\).
\end{thm}
\begin{proof}[Proof idea]
Take~\(m_{\alpha,F}(x)\in F[x]\) monic of minimal degree with \(m_{\alpha,F}(\alpha)=0\). The division and uniqueness follow by division with remainder in~\(F[x]\).
\end{proof}
\begin{app}
Structure of simple algebraic extensions, irreducibility criterion.
\end{app}

\begin{thm}[Finite simple extensions]
The extension \(F(\alpha)/F\) is finite iff \(\alpha\)~is algebraic over~\(F\), in which case
\[[F(\alpha):F]=\deg m_{\alpha,F}(x)=\deg_F\alpha\]
\end{thm}
\begin{proof}[Proof idea]
For the forward direction, by linear dependence. The \(n+1\) elements \(1,\alpha,\ldots,\alpha^n\)
are linearly dependent, so there are nonzero \(a_0,\ldots,a_n\in F\) such that
\[a_0+a_1\alpha+\cdots+a_n\alpha^n=0\]
Therefore \(\alpha\)~is algebraic over~\(F\).

For the reverse direction and degree equality, by the structure of simple algebraic extensions, since \(F(\alpha)\iso F[x]/(m_{\alpha,F}(x))\).
\end{proof}
\begin{app}
Relating finite and (finitely generated) algebraic extensions.
\end{app}
\begin{cor}
A finite extension is algebraic, and the degree of any element is at most the degree of the extension.
\end{cor}
\begin{proof}[Proof idea]
If \(K/F\)~is finite, then for any \(\alpha\in K\), \(F(\alpha)/F\)~is finite and
\begin{equation*}
\deg_F\alpha=[F(\alpha):F]\le[K:F]\qedhere
\end{equation*}
\end{proof}

\begin{thm}[Finite extensions]
An extension~\(K/F\) is finite iff \(K=F(\alpha_1,\ldots,\alpha_n)\) where \(\alpha_1,\ldots,\alpha_n\) are algebraic over~\(F\), in which case
\[[K:F]\le\prod_{i=1}^n\deg_F\alpha_i\]
\end{thm}
\begin{proof}[Proof idea]
For the forward direction, let \(\alpha_1,\ldots,\alpha_n\) be a basis for~\(K\) over~\(F\).

For the reverse direction, by induction using the tower law. Note the result holds in the simple case \(n=1\), and if \(n>1\),
\[K=F(\alpha_1,\ldots,\alpha_n)=F(\alpha_1,\ldots,\alpha_{n-1})(\alpha_n)\]
so by induction \(L=F(\alpha_1,\ldots,\alpha_{n-1})/F\) is finite and
\begin{equation*}
[K:F]=[K:L][L:F]\le\bigl(\deg_F\alpha_n\bigr)\bigl(\prod_{i=1}^{n-1}\deg_F\alpha_i\bigr)=\prod_{i=1}^n\deg_F\alpha_i\qedhere
\end{equation*}
\end{proof}
\begin{rmk}
This result generalizes the result that a \emph{simple} extension is finite iff it is algebraic by showing that an \emph{arbitrary} extension is finite iff it is finitely generated algebraic. An algebraic extension need not be finite (for example, consider the field of real algebraic numbers).
\end{rmk}
\begin{cor}[Algebraic number fields]
If \(K/F\)~is arbitrary, the elements in~\(K\) algebraic over~\(F\) form a subfield of~\(K\).
\end{cor}
\begin{proof}[Proof idea]
If \(\alpha,\beta\) are algebraic over~\(F\), then \(F(\alpha,\beta)\)~is finite and hence algebraic over~\(F\) and contains \(\alpha\pm\beta\), \(\alpha\beta\), and \(\alpha/\beta\) (if \(\beta\ne0\)).
\end{proof}
\begin{cor}[Algebraic extensions]
If \(K\)~is algebraic over~\(L\) and \(L\)~is algebraic over~\(F\), then \(K\)~is algebraic over~\(F\).
\end{cor}
\begin{proof}[Proof idea]
If \(\alpha\in K\), \(\alpha\)~is the root of some polynomial \(p(x)=a_nx^n+\cdots+a_0\in L[x]\). Now
\[F\subseteq F(a_0,\ldots,a_n)\subseteq F(a_0,\ldots,a_n)(\alpha)\]
Each extension is finite, so by the tower law the full extension is finite and algebraic, so \(\alpha\)~is algebraic over~\(F\).
\end{proof}

\begin{thm}[Existence of splitting fields]
If \(f(x)\in F[x]\) is nonconstant, there exists a splitting field for~\(f(x)\) over~\(F\).
\end{thm}
\begin{proof}[Proof idea]
By induction on the degree of~\(f(x)\), using existence of simple algebraic extensions.
\end{proof}
\begin{app}
Adjoining all roots of a polynomial.
\end{app}

\begin{thm}[Uniqueness of splitting fields]
Let \(\varphi:F\to F^*\) be an isomorphism, \(f(x)\in F[x]\) be nonconstant, and \(f^*(x)\in F^*[x]\) correspond to~\(f(x)\) under~\(\varphi\). If \(E\)~is any splitting field for~\(f(x)\) over~\(F\) and \(E^*\)~is any splitting field for~\(f^*(x)\) over~\(F^*\), then \(\varphi\)~extends to an isomorphism \(\Phi:E\to E^*\):
\begin{center}
\begin{tabular}{rlll}
\(\Phi:\)&\(E\)&\(\longrightarrow\)&\(E^*\)\\
&\(\;\Bigl|\)&&\(\;\Bigr|\)\\
\(\varphi:\)&\(F\)&\(\longrightarrow\)&\(F^*\)
\end{tabular}
\end{center}
In particular, the splitting field for~\(F\) is unique up to isomorphism.
\end{thm}
\begin{proof}[Proof idea]
By induction on the degree of~\(f(x)\), using uniqueness of simple algebraic extensions.
\end{proof}
\begin{app}
Computing in familiar extensions, showing that separability is not an embedding property.
\end{app}

\begin{thm}[Existence of algebraic closures]
\(F\)~has an algebraic closure.
\end{thm}
\begin{proof}[Proof idea]
Construct an algebraically closed field containing~\(F\) one step at a time in countably many steps, at each step adjoining a new element for each nonconstant polynomial over the current field and taking a quotient to impose root relations, and then taking a union at the limit step. Finally, take the set of elements in the union which are algebraic over~\(F\).

Let \(F_0=F\) and for each \(i\ge0\) recursively define~\(F_{i+1}\) from~\(F_i\) as follows: introduce distinct indeterminates~\(x_f\) for each nonconstant \(f(x)\in F_i[x]\), and set
\[F_{i+1}=F_i[\ldots,x_f,\ldots]/M\]
where \(M\)~is a maximal ideal containing the proper ideal \(I=(\ldots,f(x_f),\ldots)\). Then \(F_{i+1}\)~is an extension of~\(F_i\) in which every nonconstant polynomial over~\(F_i\) has a root. Take the union
\[K=\bigunion_{i=0}^{\infty} F_i\]
Then \(K\)~is algebraically closed since any nonconstant polynomial over~\(K\) has all its coefficients lying in some~\(F_i\), and hence a root lying in \(F_{i+1}\subseteq K\). The set of elements in~\(K\) algebraic over~\(F\) is the closure of~\(F\).
\end{proof}

\begin{thm}[Uniqueness of algebraic closures]
The algebraic closure of~\(F\) is unique up to isomorphism.
\end{thm}

\begin{thm}[Fundamental theorem of algebra]
\(\C\)~is algebraically closed.
\end{thm}

\begin{rmk}
The previous two results show that when working with algebraic numbers over~\(\Q\), we might as well work in~\(\C\).
\end{rmk}

\begin{thm}[Separability and derivatives]
A nonconstant polynomial \(f(x)\in F[x]\) is separable iff \(f(x)\)~is relatively prime to its formal derivative~\(f'(x)\).
\end{thm}
\begin{proof}[Proof idea]
By direct computation.
\end{proof}
\begin{app}
Determining separability without going into the splitting field.
\end{app}

\begin{thm}[Separability]
If either \(\charsubgroupeq F=0\), or \(\charsubgroupeq F=p\) and \(F\)~is finite, then a nonconstant polynomial over~\(F\) is separable iff it is the product of distinct irreducible factors over~\(F\).
\end{thm}
\begin{proof}[Proof idea]
If \(\charsubgroupeq F=0\), then by the formal derivative criterion any irreducible is separable, and distinct irreducibles never share a root (by the minimal polynomial).

If \(\charsubgroupeq F=p\) and \(F\)~is finite, suppose towards a contradiction that \(f(x)\in F[x]\) is irreducible but inseparable. By the formal derivative criterion, \(f'(x)=0\). Since \(\charsubgroupeq F=p\), \(f(x)=q(x^p)\) for some \(q(x)\in F[x]\). Since \(F\)~is finite, by the Frobenius property every coefficient of~\(q(x)\) has a \(p\)-th root and \(f(x)\)~is actually a \(p\)-th power---a contradiction.
\end{proof}
\begin{app}
Determining separability without going into the splitting field.
\end{app}

\subsection*{Techniques}
\begin{itemize}[itemsep=0pt]
\item Relating ideal structure to global structure.
\item Induction on degree of polynomials.
\item Division with remainder of polynomials.
\item Viewing all fields as extension fields.
\item Using linear algebra (module theory) techniques to study field extensions.
\item Using the tower law to determine structure of field extensions (cf. tower law for groups).
\item Adjoining `free' elements to fields and taking quotients by maximal ideals to construct extension fields satisfying desired relations (cf. free groups and group presentations).
\item Using results about finite extensions to derive results about arbitrary algebraic extensions, by working with only finitely many elements at a time.
\item Proceeding one step at a time in countably many steps.
\item Doing computations involving algebraic elements in one large algebraically closed field.
\item Using the minimal polynomial as an irreducibility criterion.
\item Computing multiplicative inverses in field extensions:
\begin{itemize}[itemsep=0pt]
\item Euclidean algorithm
\item Plugging and chugging into the minimal polynomial
\end{itemize}
\end{itemize}
