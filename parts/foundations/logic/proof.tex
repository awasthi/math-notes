%
% Notes on Mathematics
% John Peloquin
%
% Foundations
% Logic
% Proof
%
\section{Proof}
\subsection*{Theorems}
Fix a symbol set~\(S\).
\begin{thm}[Correctness]
If \(\Phi\proves\varphi\), then \(\Phi\models\varphi\).	
\end{thm}
\begin{proof}[Proof idea]
Induction on derivable sequents (in the sequent calculus for~\(S\)).
\end{proof}
\begin{app}
Showing that formal proofs do not yield incorrect results.
\end{app}
\begin{cor}[Correctness]
If \(\Phi\)~is satisfiable, then \(\Phi\)~is consistent.
\end{cor}
\begin{proof}[Proof idea]
If \(\Phi\)~is inconsistent, then \(\Phi\proves\varphi\) and \(\Phi\proves\lnot\varphi\) for some~\(\varphi\). But then \(\Phi\models\varphi\) and \(\Phi\models\lnot\varphi\) by the theorem, so \(\Phi\)~is not satisfiable.
\end{proof}

\begin{rmk}
Note the corollary is actually equivalent to the theorem. Indeed, if a set of formulas is consistent whenever it is satisfiable, then if \(\Phi\proves\varphi\), the set \(\Phi\union\{\lnot\varphi\}\) is not consistent and hence not satisfiable, so \(\Phi\models\varphi\).
\end{rmk}

\begin{thm}[Finiteness]
\ 
\begin{enumerate}[itemsep=0pt]
\item[(a)] \(\Phi\proves\varphi\) iff \(\Phi_0\proves\varphi\) for some finite \(\Phi_0\subseteq\Phi\).
\item[(b)] \(\Phi\)~is consistent iff \(\Phi_0\)~is consistent for all finite \(\Phi_0\subseteq\Phi\).
\item[(c)] \(\Phi\)~is inconsistent iff \(\Phi_0\)~is inconsistent for some finite \(\Phi_0\subseteq\Phi\).
\end{enumerate}
\end{thm}
\begin{proof}[Proof idea]
By definition of proof.
\end{proof}
\begin{app}
With correctness and completeness, yields compactness.
\end{app}
