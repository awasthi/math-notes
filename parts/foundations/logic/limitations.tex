%
% Notes on Mathematics
% John Peloquin
%
% Foundations
% Logic
% Limitations of the Formal Method
\section{Limitations of the Formal Method}
\subsection*{Theorems}
\begin{lem}
Let \(T\)~be a first-order theory.
\begin{enumerate}[itemsep=0pt]
\item[(a)] If \(T\)~is recursively axiomatizable, \(T\)~is recursively enumerable.
\item[(b)] If \(T\)~is recursively enumerable and complete, then \(T\)~is decidable.
\end{enumerate}
\end{lem}
\begin{proof}[Proof idea]
For~(a), write \(T=\Phi^{\proves}\) where \(\Phi\)~is a recursive set of axioms. Systematically generate all possible sequent derivations, and use the decision procedure for~\(\Phi\) to decide for each one whether it is a derivation from~\(\Phi\). List only the consequences derivable from~\(\Phi\).

For~(b), to decide whether \(\varphi\in T\), enumerate~\(T\) until either \(\varphi\)~or~\(\lnot\varphi\) is found.
\end{proof}

\begin{thm}[Halting problem]
The set of programs which halt on an empty input is undecidable.
\end{thm}
\begin{proof}[Proof idea]
Diagonalization.

First, choose an effective coding of programs as strings over a finite alphabet. Then argue that the set of programs which halt on their own coding is undecidable, lest we could diagonalize and construct a program different from every program. Now reduce this case to the desired case.
\end{proof}
\begin{app}
Reducing to other problems to establish undecidability.
\end{app}

\begin{thm}[Undecidability of first-order logic (Church)]
The first-order validities are undecidable.
\end{thm}
\begin{proof}[Proof idea]
Reduce the halting problem to this problem.

For a given program, choose an appropriate (effectively given) symbol set and effectively construct a sentence which describes the operation and halting of the program (on the empty input). Then the program halts iff the sentence is valid, so the validities must be undecidable.

In more detail, given a program~\(P\) with instructions \(\alpha_0,\ldots,\alpha_k\) and using at most \(r\)~registers, choose a symbol set which is suitable for representing ordered numerals and contains an \((r+2)\)-ary relation symbol~\(R\). Use \(R\)~to describe the states reached by~\(P\) under its computation on the empty input (where a \emph{state} consists of a step, instruction pointer, and register dump, all represented as numerals). Construct a sentence
\[\varphi_P=(\psi_E\land\psi_I\land\psi_{\alpha_0}\land\cdots\land\psi_{\alpha_k})\limp\psi_H\]
where \(\psi_E\)~asserts basic properties of the ordering, \(\psi_I\)~describes the initial state, \(\psi_{\alpha_i}\)~describes the state transitions induced by instruction~\(\alpha_i\), and \(\psi_H\)~asserts the existence of a halting state. Then \(\varphi_P\)~is valid iff \(P\)~halts on the empty input.
\end{proof}

\begin{thm}[Trahtenbrot]
The first-order finite validities are not recursively enumerable.
\end{thm}
\begin{proof}[Proof idea]
Reduce the halting problem to this problem.

The set~\(\fs\) of finitely \emph{satisfiable} sentences is recursively enumerable, since it is possible to effectively generate all first-order sentences and all finite models. But \(\fs\)~is not decidable, lest it is possible to decide the halting problem by making use of the antecedent of~\(\varphi_P\) from the last proof. Now \(\varphi\not\in\fs\) iff \(\lnot\varphi\in\fv\), so \(\fv\)~cannot be recursively enumerable.
\end{proof}
\begin{app}
Reducing to other problems, for example to show that the validities in stronger logical systems (where finiteness can be characterized somehow) are not recursively enumerable. See the incompleteness of second-order logic (below), Lindstr\"om's Second Theorem.
\end{app}

\begin{thm}[Incompleteness of second-order logic (G\"odel)]
The second-order validities are not recursively enumerable.
\end{thm}
\begin{proof}[Proof idea]
Reduce the first-order finite validities to the second-order validities.

Let \(\varphi_{\mathrm{fin}}\)~be a second-order sentence characterizing the finite models. Then for a first-order sentence~\(\psi\), \(\varphi_{\mathrm{fin}}\limp\psi\) is a second-order validity iff \(\psi\)~is a first-order finite validity. The result follows from Trahtenbrot's theorem.
\end{proof}
\begin{app}
We know that there is no correct and complete sequent calculus for second-order logic in general. This result shows that there is no such calculus for even the validities.
\end{app}

\begin{thm}[Undecidability of arithmetic]
The first-order theory of ordinary arithmetic is undecidable.
\end{thm}
\begin{proof}[Proof idea]
Reduce the halting problem to this problem, as with the proof of the undecidability of first-order logic. The only new difficulty is talking about sequences of numbers within arithmetic, which can be overcome using an effective coding (e.g. a prime coding, \(p\)-adic coding, etc.).
\end{proof}
\begin{rmk}
This proof demonstrates that ordinary arithmetic \emph{allows representations} of decidable predicates. This is also true for the axiomatic system of Peano arithmetic.
\end{rmk}

\begin{cor}
Ordinary arithmetic is not recursively enumerable, and so not recursively axiomatizable. Equivalently, for any recursive axiomatization of ordinary arithmetic, there exist true statements which cannot be proved.
\end{cor}
\begin{proof}[Proof idea]
Since the theory is complete, this follows from the lemma above.
\end{proof}

\noindent For the following theorems, we work in the language~\(L^{S_{\mathrm{ar}}}\) of first-order arithmetic. Fix an effective, bijective coding (G\"odel numbering) of formulas by natural numbers, and let \(\code{n}{\varphi}\)~denote the code of~\(\varphi\).

\begin{thm}[Fixed point theorem (Carnap)]
Let \(\Phi\)~be a set of sentences which allows representations. For every formula~\(\psi(x)\), there exists a sentence~\(\varphi\) such that
\[\Phi\proves\varphi\liff\psi(\code{n}{\varphi})\]
\end{thm}
\begin{proof}[Proof idea]
Diagonalization.

For notational convenience, identify formulas with their code numbers. Now let \(\varphi_0(x),\varphi_1(x),\ldots\) be an enumeration of formulas with one free variable~\(x\). For each~\(k\), consider the enumeration
\[\mathbf{E}_k:\quad\varphi_k(\varphi_0)\quad\varphi_k(\varphi_1)\quad\cdots\quad\varphi_k(\varphi_k)\quad\cdots\]
of sentences obtained by syntactically substituting formulas (code numbers) into~\(\varphi_k\). Then the diagonal enumeration is given by
\[\mathbf{D}:\quad\varphi_0(\varphi_0)\quad\varphi_1(\varphi_1)\quad\cdots\quad\varphi_k(\varphi_k)\quad\cdots\]
Substitute the sentences (code numbers) on~\(\mathbf{D}\) into~\(\psi\) to obtain the enumeration
\[\mathbf{D}^*:\quad\psi(\varphi_0(\varphi_0))\quad\psi(\varphi_1(\varphi_1))\quad\cdots\quad\psi(\varphi_k(\varphi_k))\quad\cdots\]
Since sentences on \(\mathbf{D}^*\)~can be obtained by effective syntactic operation, and \(\Phi\)~allows representations, it can be argued that \(\mathbf{D}^*\)~is `equivalent' to some~\(\mathbf{E}_k\). In other words, for some~\(k\), for all~\(x\), \(\Phi\proves\varphi_k(\varphi_x)\liff\psi(\varphi_k(\varphi_x))\). Now \(\mathbf{D}\)~and~\(\mathbf{E}_k\) have their \((k+1)\)-th entry in common, hence we obtain a fixed point at \(x=k\) by letting \(\varphi=\varphi_k(\varphi_k)\). Then \(\Phi\proves\varphi\liff\psi(\varphi)\) as desired.
\end{proof}
\begin{app}
Self-referential statements. Tarski's theorem. G\"odel's theorems.
\end{app}

\begin{thm}[Tarski]
Let \(\Phi\)~be a consistent set of sentences which allows representations. Then \(\Phi\)~cannot represent its own theory.
\end{thm}
\begin{proof}[Proof idea]
Suppose \(\chi\)~represents the theory, so for all sentences~\(\varphi\), if \(\Phi\models\varphi\) then \(\Phi\proves\chi(\code{n}{\varphi})\), and if not \(\Phi\models\varphi\) then \(\Phi\proves\lnot\chi(\code{n}{\varphi})\). Let \(\varphi\)~be a fixed point of \(\psi=\lnot\chi\) (so intuitively \(\varphi\)~states `I am not true'). Then
\[\Phi\models\varphi\iff\Phi\proves\varphi\iff\Phi\proves\lnot\chi(\code{n}{\varphi})\iff\text{ not }\Phi\models\varphi\]
---a contradiction.
\end{proof}

\begin{cor}
Ordinary arithmetic cannot represent its own theory.
\end{cor}

\begin{thm}[First incompleteness theorem (G\"odel)]
Let \(\Phi\)~be a consistent, recursive set of sentences which allows representations. Then the theory of~\(\Phi\) is incomplete, that is, there exists a sentence~\(\varphi\) such that not \(\Phi\proves\varphi\) and not \(\Phi\proves\lnot\varphi\).
\end{thm}
\begin{proof}[Proof idea]
If the theory of~\(\Phi\) is complete, then it is decidable (by the lemma above). But then since \(\Phi\)~allows representations, \(\Phi\)~can represent its theory, contradicting Tarski's theorem (in syntactic form).
\end{proof}

\begin{thm}[Second incompleteness theorem (G\"odel)]
Let \(\Phi\)~be a consistent, recursive set of sentences extending Peano arithmetic. Then \(\Phi\)~cannot prove its own consistency.
\end{thm}
\begin{proof}[Proof idea]
Since \(\Phi\)~is recursive and allows representations (since Peano arithmetic does), \(\Phi\)~can represent derivability from~\(\Phi\). That is, there exists a formula~\(\delta(x)\) such that for all sentences~\(\varphi\), if \(\Phi\proves\varphi\), then \(\Phi\proves\delta(\code{n}{\varphi})\). (This does \emph{not} mean that \(\Phi\)~can represent its theory!) Note \(\Phi\)~is consistent iff not \(\Phi\proves\lnot0\equiv 0\), hence the sentence \(\varphi_{\mathrm{con}}=\lnot\delta(\code{n}{\lnot0\equiv0})\) expresses the consistency of~\(\Phi\).

Now let \(\varphi\)~be a fixed point for \(\psi=\lnot\delta\) (so intuitively \(\varphi\)~states `I am not provable'). Since \(\Phi\)~is consistent, not \(\Phi\proves\varphi\). It can be shown that
\[\Phi\proves\varphi_{\mathrm{con}}\limp\lnot\delta(\code{n}{\varphi})\]
Therefore if \(\Phi\proves\varphi_{\mathrm{con}}\), then \(\Phi\proves\lnot\delta(\code{n}{\varphi})\). But then since \(\varphi\)~is a fixed point, \(\Phi\proves\varphi\)---a contradiction.
\end{proof}
\begin{app}
Derailing Hilbert's program.
\end{app}
