%
% Notes on Mathematics
% John Peloquin
%
% Foundations
% Computability
% Effective Operations on Partial Functions
%
\section{Effective Operations on Partial Functions}
\subsection*{Theorems}
\begin{thm}
Let \(\Phi:\FF_m\to\FF_n\) be an operator. Then \(\Phi\)~is recursive iff
\begin{enumerate}[itemsep=0pt]
\item[(a)] \(\Phi\)~is continuous (and hence monotone), that is, for all \(f\in\FF_m\) and \(\vec{x},y\),
\[\Phi(f)(\vec{x})\simeq y\iff\text{there exists a finite }\theta\subseteq f\text{ with }\Phi(\theta)(\vec{x})\simeq y\]
\item[(b)] the function \(\phi(z,\vec{x})\) given by
\[\phi(z,\vec{x})\simeq\begin{cases}
\Phi(\theta)(\vec{x})&\text{if }z=\coding{\theta}\text{ for finite }\theta\in\FF_m\\
\text{undefined}&\text{otherwise}
\end{cases}\]
is computable.
\end{enumerate}
\end{thm}
\begin{app}
Easily prove operators to be recursive.
\end{app}

\begin{thm}[Myhill-Sheperdson I]
Let \(\Psi:\FF_m\to\FF_n\) be a recursive operator. Then there exists a total computable function~\(h\) such that for all~\(e\),
\[\Psi(\phi_e^{(m)})=\phi_{h(e)}^{(n)}\]
\end{thm}
\begin{proof}[Proof idea]
Proceed by search.

Let \(\psi(z,\vec{x})\) witness recursiveness of~\(\Psi\). Using the \smn\ theorem together with universal programs, construct \(\phi_{h(e)}^{(n)}\) to search for finite \(\theta\subseteq\phi_e^{(m)}\) with \(\psi(\coding{\theta},\vec{x})\)~defined, and return such a value if found.
\end{proof}

\noindent Call a total function~\(h\) a \emph{extensional} (for \(m\)~and~\(n\)) if \(\phi_a^{(m)}=\phi_b^{(m)}\) implies \(\phi_{h(a)}^{(n)}=\phi_{h(b)}^{(n)}\).

\begin{thm}[Myhill-Sheperdson II]
Let \(h\)~be an extensional total computable function. Then there exists a unique continuous operator \(\Psi:\FF_m\to\FF_n\) such that, for all~\(e\),
\[\Psi(\phi_e^{(m)})=\phi_{h(e)}^{(n)}\]
Moreover, \(\Psi\)~is recursive.
\end{thm}
\begin{proof}[Proof idea]
Proceed by extension.

First note that \(h\)~naturally induces an operation on computable functions, which includes in particular the finite functions. Any continuous operator~\(\Psi\) extending this operation must be defined in the obvious way in terms of the finite functions, and hence is unique (if it exists). By the Rice-Shapiro theorem, the operation on computable functions is continuous, and it follows that \(\Psi\)~is well defined. Now \(\Psi\)~is trivially continuous by definition, and its operation on finite functions can easily be computed using~\(h\), so \(\Psi\)~is recursive.
\end{proof}
\begin{app}
The Myhill-Sheperdson theorem establishes the equivalence of two formalizations of the notion of an effective operation on computable functions. It thereby allows one to conveniently switch between the two as needed.
\end{app}

\begin{thm}[Kleene's first recursion theorem]
Let \(\Phi:\FF_m\to\FF_n\) be a continuous operator. Then there exists a function~\(f_{\Phi}\) which is a least fixed point for~\(\Phi\), that is,
\begin{enumerate}[itemsep=0pt]
\item[(a)] \(\Phi(f_{\Phi})=f_{\Phi}\)
\item[(b)] \(\Phi(g)=g\) implies \(f_{\Phi}\subseteq g\)
\end{enumerate}
Moreover, if \(\Phi\)~is recursive, then \(f_{\Phi}\)~is computable.
\end{thm}
\begin{proof}[Proof idea]
Construct the fixed point by iterating~\(\Phi\) and taking a limit. Specifically, recursively (in the set theoretic sense) define the sequence
\begin{align*}
f_0&=f_{\emptyset}\\
f_{n+1}&=\Phi(f_n)\\
f_{\Phi}&=\bigunion_{n\ge0} f_n
\end{align*}
If \(\Phi\)~is recursive, then by the Myhill-Sheperdson theorem (part I) we can effectively index the functions in this sequence, so values of \(f_{\Phi}\)~are computable.
\end{proof}
\begin{app}
Proving existence and computability of a very broad class of recursive functions. In the semantics of formal programming languages, giving meaning to recursively defined programs.
\end{app}
\begin{rmk}
Note that the proof of the recursion theorem relies on a basic set-theoretic recursion principle to ensure the existence of the sequence~\((f_n)\), and so ultimately the existence of~\(f_{\Phi}\).

Note also the method used to construct the fixed point works in any context where there is a monotone operator behaving appropriately at limit points, and is used in other areas of mathematics.
\end{rmk}
