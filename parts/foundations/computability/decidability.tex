%
% Notes on Mathematics
% John Peloquin
%
% Foundations
% Computability
% Decidability and Partial Decidability
%
\section{Decidability and Partial Decidability}
\subsection*{Theorems}
\begin{thm}[Algebra of decidability]
The class of decidable predicates is closed under negation, conjunction, disjunction, and bounded quantification; it is not closed under unbounded quantification.
\end{thm}

\begin{thm}[Halting problem]
The predicate `\(y\in W_x\)' is undecidable.
\end{thm}
\begin{proof}[Proof idea]
Use diagonalization to prove `\(x\in W_x\)' undecidable, then reduce.
\end{proof}
\begin{app}
The halting problem shows that there is no general effective method to decide whether a program halts on a given input. It (or, more commonly, the problem `\(x\in W_x\)') is often used to prove undecidability of other problems through reduction.
\end{app}

\begin{thm}[Rice]
If \(\emptyset\subset\A\subset\CC_1\), the problem `\(\phi_x\in\A\)' is undecidable.
\end{thm}
\begin{proof}[Proof idea]
Use the \smn\ theorem to reduce \(x\in W_x\) to \(\phi_x\in\A\).
\end{proof}
\begin{app}
Establishing undecidability of many classes of computable functions.
\end{app}

\begin{thm}[Partial decidability]
The following are equivalent:
\begin{enumerate}[itemsep=0pt]
\item[(a)] \(P(\vec{x})\) is partially decidable
\item[(b)] There exists a computable function~\(f(\vec{x})\) such that \(P(\vec{x})\iff\vec{x}\in\domain(f)\)
\item[(c)] There exists a primitive recursive predicate \(R(\vec{x},y)\) such that \(P(\vec{x})\iff\exists y R(\vec{x},y)\)
\item[(d)] (Matiyasevich's theorem) \(P(\vec{x})\) is diophantine
\end{enumerate}
\end{thm}
\begin{proof}[Proof idea]
Trivially (a)\(\iff\)(b). For (b)\(\implies\)(c), set \(R(\vec{x},y)\equiv H_n(e,\vec{x},y)\) where \(f=\phi_e\), and for (c)\(\implies\)(b) set \(f(\vec{x})=\mu y\,R(\vec{x},y)\).
\end{proof}

\begin{app}
Matiyasevich's theorem solves Hilbert's tenth problem (answering in the negative by establishing the existence of undecidable diophantine predicates).
\end{app}

\begin{thm}[Algebra of partial decidability]\ 
\begin{enumerate}[itemsep=0pt]
\item[(a)] The class of partially decidable predicates is closed under conjunction, disjunction, existential quantification, and bounded universal quantification; it is not closed under negation or unbounded universal quantification.
\item[(b)] \(P(\vec{x})\) is decidable iff both \(P(\vec{x})\) and \(\lnot P(\vec{x})\) are partially decidable.
\end{enumerate}
\end{thm}
\begin{proof}[Proof idea]
For (a), closure under existential quantification follows from part~(c) of the previous theorem and searching for pairs; the rest is straightforward (consult exercises below).

For (b), use multitasking.
\end{proof}

\begin{thm}
The function~\(f(\vec{x})\) is computable iff \(f(\vec{x})\simeq y\) is partially decidable.
\end{thm}
