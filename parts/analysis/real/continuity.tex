%
% Notes on Mathematics
% John Peloquin
%
% Analysis
% Real Analysis
% Continuity
%
\section{Continuity}
\subsection*{Definitions}
\begin{defn}
Let \(X,Y\) be metric spaces, \(E\subseteq X\), and \(f:E\to Y\). Let \(p\in X\) be a limit point of~\(E\) and \(q\in Y\). If for every \(\epsilon>0\) there exists \(\delta>0\) such that \(d_Y(f(x),q)<\epsilon\) for all \(x\in E\) with \(0<d_X(x,p)<\delta\), then \(q\)~is a \emph{limit} of~\(f\) at~\(p\), denoted \(q=\lim_{x\to p}f(x)\) or \(f(x)\to q\) as \(x\to p\).
\end{defn}

\begin{defn}
Let \(X,Y\) be metric spaces, \(E\subseteq X\), and \(f:E\to Y\). Let \(p\in E\). Then \(f\)~is \emph{continuous at~\(p\)} if for every \(\epsilon>0\) there exists \(\delta>0\) such that \(d_Y(f(x),f(p))<\epsilon\) for all \(x\in E\) with \(d_X(x,p)<\delta\).

If \(f\)~is not continuous at~\(p\) (but defined at~\(p\)), \(f\)~is \emph{discontinuous at~\(p\)}.

If \(f\)~is continuous at every point of~\(E\), then \(f\)~is \emph{continuous (on~\(E\))}.
\end{defn}

\begin{defn}
Let \(X,Y\) be metric spaces, \(E\subseteq X\), and \(f:E\to Y\). Then \(f\)~is \emph{uniformly continuous (on~\(E\))} if for every \(\epsilon>0\), there exists \(\delta>0\) such that \(d_Y(f(x),f(y))<\epsilon\) for all \(x,y\in E\) with \(d_X(x,y)<\delta\).
\end{defn}

\begin{defn}
Let \(Y\)~be a metric space and \(f:(a,b)\to Y\). Let \(p\in(a,b)\) and \(q\in Y\). If for every \(\epsilon>0\) there exists \(\delta>0\) such that \(d_Y(f(x),q)<\epsilon\) for all \(x\in(p-\delta,p)\), then \(q\)~is a \emph{left-hand limit} of~\(f\) at~\(p\), denoted \(q=f(p-)\). Analogously for \emph{right-hand limit}~\(f(p+)\).
\end{defn}

\subsection*{Theorems}
\begin{thm}[Characterization of limits by sequences]
Let \(X,Y\) be metric spaces, \(E\subseteq X\), and \(f:E\to Y\). Let \(p\in X\) be a limit point of~\(E\) and \(q\in Y\). Then
\begin{align*}
&\lim_{x\to p}f(x)=q&&\iff&&\lim_{n\to\infty}f(x_n)=q\\
&&&&&\text{for all sequences }\{x_n\}\text{ in }E\\
&&&&&\text{with }\lim x_n=p\text{ and }x_n\ne p
\end{align*}
\end{thm}
\begin{proof}[Proof idea]
Use epsilons.
\end{proof}

\begin{cor}
Limits of functions are unique.
\end{cor}
\begin{proof}[Proof idea]
By uniqueness of sequential limits.
\end{proof}
\begin{rmk}
Uniqueness also holds for extended limits of functions on~\(\R\).
\end{rmk}

\begin{thm}[Limits and field operations in~\(\C\)]
Let \(X\)~be a metric space, \(E\subseteq X\), \(f:E\to\C\), and \(g:E\to\C\). If \(f\)~and~\(g\) have limits at \(p\in X\), then
\begin{enumerate}[itemsep=0pt]
\item[(a)] \(\lim_{x\to p}(f+g)(x)=\lim_{x\to p}f(x)+\lim_{x\to p}g(x)\)
\item[(b)] \(\lim_{x\to p}(fg)(x)=\lim_{x\to p}f(x)\cdot\lim_{x\to p}g(x)\)
\item[(c)] \(\lim_{x\to p}(f/g)(x)=\lim_{x\to p}f(x)/\lim_{x\to p}g(x)\) if \(\lim_{x\to p}g(x)\ne0\).
\end{enumerate}
\end{thm}
\begin{proof}[Proof idea]
By the above theorem and properties of sequential limits in~\(\C\).
\end{proof}

\begin{rmk}
Analogous results hold for extended limits of functions on~\(\R\), wherever the expressions on the right are defined (determinate forms).
\end{rmk}

\begin{thm}[Limits and operations in~\(\R^k\)]
Let \(X\)~be a metric space, \(E\subseteq X\), \(\vec{f}:E\to\R^k\), and \(\vec{g}:E\to\R^k\). If \(\vec{f}\)~and~\(\vec{g}\) have limits at \(p\in X\) and \(\alpha\in\R\), then
\begin{enumerate}[itemsep=0pt]
\item[(a)] \(\lim_{x\to p}(\alpha\vec{f})(x)=\alpha\lim_{x\to p}\vec{f}(x)\)
\item[(b)] \(\lim_{x\to p}(\vec{f}+\vec{g})(x)=\lim_{x\to p}\vec{f}(x)+\lim_{x\to p}\vec{g}(x)\)
\item[(c)] \(\lim_{x\to p}(\vec{f}\dotprod\vec{g})(x)=\lim_{x\to p}\vec{f}(x)\dotprod\lim_{x\to p}\vec{g}(x)\)
\end{enumerate}
\end{thm}
\begin{proof}[Proof idea]
By the above theorem and properties of sequential limits in~\(\R^k\).
\end{proof}

\begin{thm}[Characterizations of continuity]
Let \(X,Y\) be metric spaces, \(E\subseteq X\), and \(f:E\to Y\). Let \(p\in E\) be a limit point of~\(E\).
\begin{enumerate}[itemsep=0pt]
\item[(a)] \(f\)~is continuous at~\(p\) iff \(\lim_{x\to p}f(x)=f(p)\).
\item[(b)] \(f\)~is continuous iff \(f^{-1}\)~takes open [closed] sets to open [closed] sets.
\end{enumerate}
\end{thm}
\begin{proof}[Proof idea]
For~(a), use epsilons.

For~(b), use epsilons for open sets, then take complements for closed sets.
\end{proof}

\begin{thm}[Continuity and composition]
Let \(f:X\to Y\) be continuous and \(g:Y\to Z\) continuous on~\(f(X)\). Then \(g\circ f\)~is continuous.
\end{thm}
\begin{proof}[Proof idea]
Use nested epsilons.
\end{proof}

\begin{thm}[Continuity and field operations in~\(\C\)]
Let \(f:X\to\C\) and \(g:X\to\C\) be continuous. Then
\begin{enumerate}[itemsep=0pt]
\item[(a)] \(f+g\)~is continuous.
\item[(b)] \(fg\)~is continuous.
\item[(c)] \(f/g\)~is continuous if \(g(x)\ne0\) on~\(X\).
\end{enumerate}
\end{thm}
\begin{proof}[Proof idea]
At isolated points there is nothing to prove. At limit points, use the limit characterization of continuity and properties of limits in~\(\C\).
\end{proof}

\begin{thm}[Continuity in~\(\R^k\)]
Let \(\vec{f}:X\to\R^k\) and \(\vec{g}:X\to\R^k\).
\begin{enumerate}[itemsep=0pt]
\item[(a)] If \(\vec{f}=(f_1,\ldots,f_k)\), then \(\vec{f}\)~is continuous iff \(f_i\)~is continuous for \(1\le i\le k\).
\item[(b)] If \(\vec{f}\)~and~\(\vec{g}\) are continuous and \(\alpha\in\R\), then
\begin{enumerate}[itemsep=0pt]
\item[(i)] \(\alpha\vec{f}\)~is continuous.
\item[(ii)] \(\vec{f}+\vec{g}\) is continuous.
\item[(iii)] \(\vec{f}\dotprod\vec{g}\) is continuous.
\end{enumerate}
\end{enumerate}
\end{thm}
\begin{proof}[Proof idea]
For~(a), at isolated points there is nothing to prove; at limit points, use the limit characterization of continuity together with the characterization of limits (of sequences) in~\(\R^k\).

For~(b), use~(a) and properties of continuity in~\(\R\).
\end{proof}

\begin{thm}[Continuity preserves compactness]
Let \(X\)~be compact and \(f:X\to Y\) continuous. Then \(f(X)\)~is compact.
\end{thm}
\begin{proof}[Proof idea]
Since \(f^{-1}\)~takes open sets to open sets, any open covering of~\(f(X)\) can be pulled back to an open covering of~\(X\), yielding a finite subcovering.
\end{proof}

\begin{cor}[Extreme value theorem in~\(\R\)]
Let \(X\)~be compact, \(f:X\to\R\) continuous. Set
\[\alpha=\inf f(X)\qquad\beta=\sup f(X)\]
Then there exist \(x,y\in X\) with \(f(x)=\alpha\) and \(f(y)=\beta\).
\end{cor}
\begin{proof}[Proof idea]
By Heine-Borel, \(f(X)\)~is closed and bounded, so \(\alpha,\beta\in f(X)\).
\end{proof}
\begin{app}
Mean value theorem, algebraic closure of~\(\C\), etc.
\end{app}

\begin{thm}[Uniform continuity]
Let \(X\)~be compact and \(f:X\to Y\) continuous. Then \(f\)~is uniformly continuous.
\end{thm}
\begin{proof}[Proof idea]
Given \(\epsilon>0\), for each point in~\(X\) fix a neighborhood whose extension witnesses continuity of~\(f\) at the point for~\(\epsilon\). By compactness, \(X\)~is covered by finitely many such neighborhoods. Let \(\delta>0\)~be a fraction of the minimum radius. It follows that uniformity holds for~\(\epsilon\) with~\(\delta\).
\end{proof}
\begin{app}
Riemann integrability of continuous functions.
\end{app}

\begin{thm}[Continuity preserves connectedness]
Let \(X\)~be connected and \(f:X\to Y\) continuous. Then \(f(E)\)~is connected.
\end{thm}
\begin{proof}[Proof idea]
Since \(f^{-1}\)~takes closed sets to closed sets.
\end{proof}

\begin{cor}[Intermediate value theorem in~\(\R\)]
Let \(f:[a,b]\to\R\) be continuous. If \(\min\{f(a),f(b)\}<y<\max\{f(a),f(b)\}\), then there exists \(x\in(a,b)\) with \(f(x)=y\).
\end{cor}
\begin{proof}[Proof idea]
By the characterization of connectedness in~\(\R\).
\end{proof}

\begin{thm}[Monotonic functions on~\(\R\)]
If \(f:(a,b)\to\R\) increases monotonically, then for \(x\in(a,b)\),
\[\sup_{a<t<x} f(t)=f(x-)\le f(x)\le f(x+)=\inf_{x<t<b}f(t)\]
If \(a<x<y<b\), then \(f(x+)\le f(y-)\).

Analogously if \(f\)~decreases monotonically.
\end{thm}
\begin{proof}[Proof idea]
Use epsilons.
\end{proof}
\begin{app}
Monotonic functions have only jump discontinuities, and at most countably many.
\end{app}

\subsection*{Techniques}
\begin{itemize}[itemsep=0pt]
\item Reducing function limits to sequence limits.
\item Proving continuity:
\begin{itemize}[itemsep=0pt]
\item Characterizations (limit, open and closed sets).
\item Closure properties (composites, sums, products, etc.).
\end{itemize}
\item Using continuity:
\begin{itemize}[itemsep=0pt]
\item Characterizations.
\item Uniform continuity.
\item Preservation of compactness.
\item Preservation of connectedness.
\end{itemize}
\item Reducing \(\R^k\)-valued functions to \(\R\)-valued (component) functions.
\end{itemize}
