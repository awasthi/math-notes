%
% Notes on Mathematics
% John Peloquin
%
% Analysis
% Real Analysis
% Differentiability
%
\section{Differentiability}

\subsection*{Definitions}
\begin{defn}
Let \(\vec{f}:[a,b]\to\R^k\) and \(x\in[a,b]\). If the limit
\[\lim_{t\to x}\frac{\vec{f}(t)-\vec{f}(x)}{t-x}\]
exists, \(\vec{f}\)~is \emph{differentiable at~\(x\)}, and the limit is denoted~\(\vec{f}'(x)\).

The function~\(\vec{f}'\) induced is called the \emph{derivative} of~\(\vec{f}\). If \(\vec{f}'\)~is defined on \(E\subseteq[a,b]\), \(\vec{f}\)~is \emph{differentiable on~\(E\)}.

This is continued with \(\vec{f}'',\vec{f}'''\), etc. and more generally~\(\vec{f}^{(n)}\) for \(n\ge 0\).
\end{defn}

\begin{defn}
Let \(X\)~be a metric space, \(f:X\to\R\), and \(p\in X\). Then \(f(p)\)~is a \emph{local maximum} of~\(f\) at~\(p\) if there exists \(\delta>0\) such that \(f(q)\le f(p)\) for all \(q\in X\) with \(d(p,q)<\delta\). Analogously for \emph{local minimum}.
\end{defn}

\subsection*{Theorems}
\begin{thm}[Differentiability implies continuity in~\(\R^k\)]
Let \(\vec{f}:[a,b]\to\R^k\). If \(x\in[a,b]\) and \(\vec{f}\)~is differentiable at~\(x\), then \(\vec{f}\)~is continuous at~\(x\).
\end{thm}
\begin{proof}[Proof idea]
Note for \(t\ne x\),
\begin{equation*}
\vec{f}(t)-\vec{f}(x)=\frac{\vec{f}(t)-\vec{f}(x)}{t-x}\cdot(t-x)\to 0\quad\text{as}\quad t\to x\qedhere
\end{equation*}
\end{proof}
\begin{rmk}
\emph{The converse is false!} For an extreme counterexample, see the Weierstrass everywhere continuous but nowhere differentiable function.
\end{rmk}

\begin{thm}[Chain rule in~\(\R\)]
Let \(f:[a,b]\to[c,d]\) and \(g:[c,d]\to\R\). If \(x\in[a,b]\) and \(f\)~is differentiable at~\(x\) and \(g\)~is differentiable at~\(f(x)\), then
\[(g\circ f)'(x)=g'(f(x))f'(x)\]
\end{thm}
\begin{proof}[Proof idea]
By differentiability of~\(f\) at~\(x\), there exists a function~\(\delta(t)\) such that
\[f(t)-f(x)=[f'(x)+\delta(t)](t-x)\]
where \(\delta(t)\to0\) as \(t\to x\) and \(\delta(x)=0\). Similarly, by differentiability of~\(g\) at \(y=f(x)\), there exists a function~\(\epsilon(u)\) such that
\[g(u)-g(y)=[g'(y)+\epsilon(u)](u-y)\]
where \(\epsilon(u)\to0\) as \(u\to y\) and \(\epsilon(y)=0\). (Note \(\epsilon\)~is continuous at~\(y\)!)

Now write \(h=g\circ f\). Then
\begin{align*}
h(t)-h(x)&=g(f(t))-g(f(x))\\
	&=[g'(f(x))+\epsilon(f(t))][f(t)-f(x)]\\
	&=[g'(f(x))+\epsilon(f(t))][f'(x)+\delta(t)](t-x)
\end{align*}
Now divide by \(t-x\) and let \(t\to x\). (Note this relies on continuity of~\(\epsilon\) at~\(y\) since it might be that \(f(t)=y\) infinitely often as \(t\to x\).)
\end{proof}

\begin{thm}[Derivatives and field operations in~\(\C\)]
Let \(f:[a,b]\to\C\) and \(g:[a,b]\to\C\) be differentiable at \(x\in[a,b]\). Then
\begin{enumerate}[itemsep=0pt]
\item[(a)] \((f+g)'(x)=f'(x)+g'(x)\)
\item[(b)] \((fg)'(x)=f'(x)g(x)+f(x)g'(x)\)
\item[(c)] \((f/g)'(x)=\displaystyle\frac{f'(x)g(x)-f(x)g'(x)}{[g(x)]^2}\quad(g(x)\ne0)\)
\end{enumerate}
\end{thm}
\begin{proof}[Proof idea]
Express difference quotients of the new functions in terms of difference quotients of the given functions, then take limits.

For~(a), this is trivial.

For~(b), similarly to sequence products note
\begin{multline*}
f(t)g(t)-f(x)g(x)=[f(t)-f(x)][g(t)-g(x)]\\
	+[f(t)-f(x)]g(x)+[g(t)-g(x)]f(x)
\end{multline*}
Now divide by \(t-x\) and let \(t\to x\).

For~(c), note
\[\frac{f(t)}{g(t)}-\frac{f(x)}{g(x)}=\frac{[f(t)-f(x)]g(x)-[g(t)-g(x)]f(x)}{g(t)g(x)}\]
Now divide by \(t-x\) and let \(t\to x\).
\end{proof}

\begin{thm}[Derivatives in~\(\R^k\)]
Let \(\vec{f}:[a,b]\to\R^k\) and \(\vec{g}:[a,b]\to\R^k\).
\begin{enumerate}[itemsep=0pt]
\item[(a)] If \(\vec{f}=(f_1,\ldots,f_k)\), \(\vec{f}\)~is differentiable at \(x\in[a,b]\) iff \(f_i\)~is differentiable at~\(x\) for \(1\le i\le k\), in which case
\[\vec{f}'(x)=(f_1'(x),\ldots,f_k'(x))\]
\item[(b)] If \(\vec{f}\)~and~\(\vec{g}\) are differentiable at \(x\in[a,b]\) and \(\alpha\in\R\), then
\begin{enumerate}[itemsep=0pt]
\item[(i)] \((\alpha\vec{f})'(x)=\alpha\vec{f}'(x)\)
\item[(ii)] \((\vec{f}+\vec{g})'(x)=\vec{f}'(x)+\vec{g}'(x)\)
\item[(iii)] \((\vec{f}\dotprod\vec{g})'(x)=\vec{f}'(x)\dotprod\vec{g}(x)+\vec{f}(x)\dotprod\vec{g}'(x)\)
\end{enumerate}
\end{enumerate}
\end{thm}
\begin{proof}[Proof idea]
For~(a), use the characterization of limits (of sequences) in~\(\R^k\).

For~(b), use~(a) and properties of derivatives in~\(\R\).
\end{proof}

\begin{thm}[Fermat]
Let \(f:[a,b]\to\R\). If \(f\)~has a local maximum or local minimum at \(x\in(a,b)\), and \(f\)~is differentiable at~\(x\), then \(f'(x)=0\).
\end{thm}
\begin{proof}[Proof idea]
Look at left-hand and right-hand limits separately.
\end{proof}
\begin{thm}[Rolle]
Let \(f:[a,b]\to\R\). If \(f(a)=f(b)\) and \(f\)~is continuous on~\([a,b]\) and differentiable on~\((a,b)\), there exists \(x\in(a,b)\) with \(f'(x)=0\).
\end{thm}
\begin{proof}[Proof idea]
By the extreme value theorem, there exists \(x\in(a,b)\) such that \(f(x)\)~is maximum or minimum for~\(f\), so \(f'(x)=0\).
\end{proof}

\begin{thm}[Mean value theorems in~\(\R\)]
Let \(f:[a,b]\to\R\) and \(g:[a,b]\to\R\) be continuous on~\([a,b]\) and differentiable on~\((a,b)\).
\begin{enumerate}[itemsep=0pt]
\item[(a)] (Cauchy) There exists \(x\in(a,b)\) with
\[f'(x)[g(b)-g(a)]=g'(x)[f(b)-f(a)]\]
\item[(b)] There exists \(x\in(a,b)\) with
\[f'(x)=\frac{f(b)-f(a)}{b-a}\]
\end{enumerate}
\end{thm}
\begin{proof}[Proof idea]
For~(a), set
\[h(x)=f(x)[g(b)-g(a)]-g(x)[f(b)-f(a)]\]
so \(h'(x)=0\) for \(x\in(a,b)\) iff (a)~holds. Now apply Rolle's theorem to~\(h\).

Now (b)~follows from~(a) with \(g(x)=x\).
\end{proof}
\begin{app}
Relating values of functions to values of derivatives, Taylor's theorem, L'Hospital's rule, change of variable for Riemann-Stieltjes integration, fundamental theorem of calculus for Riemann integration, etc.
\end{app}
\begin{rmk}
Note the mean value theorem asserts the existence of a point where the instantaneous rate of change of a differentiable function equals its average rate of change (on an interval).
\end{rmk}
\begin{rmk}
\emph{The mean value theorem fails in~\(\R^k\)!} But see the mean value inequality.
\end{rmk}

\begin{cor}[Monotonicity in~\(\R\)]
Let \(f:(a,b)\to\R\) be differentiable.
\begin{enumerate}[itemsep=0pt]
\item[(a)] \(f\)~is monotonically increasing iff \(f'(x)\ge0\) for all \(x\in(a,b)\).
\item[(b)] \(f\)~is constant iff \(f'(x)=0\) for all \(x\in(a,b)\).
\item[(c)] \(f\)~is monotonically decreasing iff \(f'(x)\le0\) for all \(x\in(a,b)\).
\end{enumerate}
\end{cor}
\begin{proof}[Proof idea]
By the mean value theorem.
\end{proof}

\begin{thm}[Taylor]
Let \(f:[a,b]\to\R\). Suppose \(n>0\) and \(f^{(n-1)}\)~is continuous on~\([a,b]\) and differentiable on~\((a,b)\). If \(\alpha,\beta\in(a,b)\) and \(\alpha\ne\beta\), there exists~\(x\) between \(\alpha,\beta\) such that
\[f(\beta)=P(\beta)+\frac{f^{(n)}(x)}{n!}(\beta-\alpha)^n\]
where
\[P(x)=\sum_{k=0}^{n-1}\frac{f^{(k)}(\alpha)}{k!}(x-\alpha)^k\]
is the Taylor polynomial of degree~\((n-1)\) for~\(f\) at~\(\alpha\).
\end{thm}
\begin{proof}[Proof idea]
By repeated application of Rolle's theorem.

Define~\(C\) using \(f(\beta)=P(\beta)+C(\beta-\alpha)^n\), so \(x\)~between~\(\alpha,\beta\) is desired satisfying \(f^{(n)}(x)=n!C\). Define
\[h(x)=f(x)-P(x)-C(x-\alpha)^n\]
so the result holds for~\(x\) iff \(h^{(n)}(x)=0\).

Note \(h^{(k)}(\alpha)=0\) for \(0\le k\le n-1\) and \(h(\beta)=0\). By Rolle's theorem, there exists~\(x_1\) between~\(\alpha,\beta\) with \(h^{(1)}(x_1)=0\); then there exists~\(x_2\) between~\(\alpha,x_1\) with \(h^{(2)}(x_2)=0\); etc. Finally, there exists~\(x_n\) with \(h^{(n)}(x_n)=0\), so the result holds with \(x=x_n\).
\end{proof}
\begin{app}
Approximating differentiable functions with error bound.
\end{app}
\begin{rmk}
Taylor's theorem generalizes the mean value theorem, which is just case \(n=1\) of Taylor's theorem.
\end{rmk}

\begin{thm}[L'Hospital]
Let \(f:(a,b)\to\R\) and \(g:(a,b)\to\R\) be differentiable with \(-\infty\le a<b\le+\infty\), where \(g'(x)\ne0\) in~\((a,b)\). Let \(a\le c\le b\). If \(f(x),g(x)\to0\) as \(x\to c\), or \(g(x)\to\pm\infty\) as \(x\to c\), and if
\[\frac{f'(x)}{g'(x)}\to A\quad\text{as}\quad x\to c\qquad(-\infty\le A\le+\infty)\]
Then \(f(x)/g(x)\to A\) as \(x\to c\).
\end{thm}
\begin{proof}[Proof idea]
By Cauchy's mean value theorem.

If \(A<+\infty\), use the theorem to relate values of the derivative quotient to values of the function quotient and show that for all~\(r\) with \(A<r\), there is some neighborhood of~\(c\) in which \(f(x)/g(x)<r\).

If \(A>-\infty\), do the inverse.
\end{proof}
\begin{app}
Evaluating limits yielding indeterminate forms such as \(\frac{0}{0}\), \(\frac{\infty}{\infty}\), etc.
\end{app}
\begin{rmk}
\emph{L'Hospital's rule fails in~\(\R^k\)!}
\end{rmk}

\begin{thm}[Intermediate value theorem for derivatives in~\(\R\)]
Let \(f:[a,b]\to\R\) be differentiable. If \(\min\{f'(a),f'(b)\}<y<\max\{f'(a),f'(b)\}\), there exists \(x\in(a,b)\) with \(f'(x)=y\).
\end{thm}
\begin{proof}[Proof idea]
Set \(h(x)=f(x)-yx\). Then \(h'(a)\)~and~\(h'(b)\) differ in sign, so by the extreme value theorem \(h\)~has a maximum or minimum at some \(x\in(a,b)\) where \(h'(x)=0\), that is, \(f'(x)=y\).
\end{proof}

\begin{thm}[Mean value inequality in~\(\R^k\)]
Let \(\vec{f}:[a,b]\to\R^k\) be continuous on~\([a,b]\) and differentiable on~\((a,b)\). Then there exists \(x\in(a,b)\) with
\[\norm{\vec{f}'(x)}\ge\frac{\norm{\vec{f}(b)-\vec{f}(a)}}{b-a}\]
\end{thm}
\begin{proof}[Proof idea]
Construct an \(\R\)-valued function from~\(\vec{f}\) using the dot product and apply the mean value theorem in~\(\R\), then use Cauchy-Schwarz.

In detail, set \(\vec{z}=\vec{f}(b)-\vec{f}(a)\), and define \(\varphi(t)=(\vec{z}\dotprod\vec{f})(t)\). By the mean value theorem applied to~\(\varphi\), there exists \(x\in(a,b)\) with
\[\vec{z}\dotprod\vec{f}'(x)=\frac{\vec{z}\dotprod\vec{f}(b)-\vec{z}\dotprod\vec{f}(a)}{b-a}=\frac{\vec{z}\dotprod\vec{z}}{b-a}=\frac{\norm{\vec{z}}^2}{b-a}\]
By Cauchy-Schwarz, \(\norm{\vec{z}\vphantom{\vec{f}'(x)}}\norm{\vec{f}'(x)}\ge\vec{z}\dotprod\vec{f}'(x)\), so
\begin{equation*}
\norm{\vec{f}'(x)}\ge\frac{\norm{z}}{b-a}=\frac{\norm{\vec{f}(b)-\vec{f}(a)}}{b-a}\qedhere
\end{equation*}
\end{proof}
\begin{app}
Bounding change in a function by values of its derivative, as in the proof of the theorem on uniform convergence and differentiation.
\end{app}

\subsection*{Techniques}
\begin{itemize}[itemsep=0pt]
\item Proving differentiability and calculating derivatives:
\begin{itemize}[itemsep=0pt]
\item Limit of difference quotient.
\item Closure properties (composites, sums, products, etc.).
\end{itemize}
\item Relating values of functions to values of their derivatives using mean value theorems.
\item Translating problems about arbitrary values of derivatives into problems about zeros of derivatives, then using extrema or mean value theorems to find zeros.
\item Approximating functions using Taylor series expansions.
\item Reducing \(\R^k\)-valued functions to \(\R\)-valued functions.
\end{itemize}
