%
% Notes on Mathematics
% John Peloquin
%
% Analysis
% Complex Analysis
% Complex Numbers
\section{Complex Numbers}
\subsection*{Definitions}
Basic definitions assumed.
\subsection*{Theorems}
\begin{thm}[Existence of~\(\C\)]
There exists an algebraically closed field~\(\C\) (the complex numbers) containing~\(\R\) as a subfield.
\end{thm}
\begin{proof}[Proof idea]
Let complex numbers be ordered pairs of real numbers with addition and multiplication defined appropriately. Prove algebraic closure later.
\end{proof}
\begin{app}
All subsequent theory.
\end{app}

\begin{thm}[Polar form]
If \(z\ne0\), there exist unique \(r\ge0\) and \(\theta\in(-\pi,\pi]\) such that
\[z=r(\cos\theta+i\sin\theta)=re^{i\theta}\]
\end{thm}
\begin{proof}[Proof idea]
By geometry in the complex plane and Euler's identity (definition).
\end{proof}
\begin{app}
Multiplication, geometry of multiplication as scaling and rotation.
\end{app}

\begin{thm}[Roots]
If \(z=re^{i\theta}\) and \(n\ge1\), the \(n\)~distinct \(n\)-th roots of~\(z\) are given by
\[w_k=\sqrt[n]{r}\exp\Bigl(\frac{\theta+2k\pi}{n}\Bigr)\quad(k=0,\ldots,n-1)\]
\end{thm}
\begin{proof}[Proof idea]
Polar form.

In detail, if \(w=se^{i\phi}\) is an \(n\)-th root of~\(z\), then
\[s^ne^{in\phi}=w^n=z=re^{i\theta}\]
so by geometry in the complex plane, \(s^n=r\) and \(n\phi=\theta+2k\pi\) for some integer~\(k\). The distinct roots are obtained by letting \(k=0,\ldots,n-1\).
\end{proof}

\subsection*{Techniques}
\begin{itemize}[itemsep=0pt]
\item Using polar form for multiplication.
\end{itemize}
