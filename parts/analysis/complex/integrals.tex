%
% Notes on Mathematics
% John Peloquin
%
% Analysis
% Complex Analysis
% Integrals
%
\section{Integrals}
\subsection*{Definitions}
\begin{defn}
An \emph{arc} (or \emph{curve} or \emph{path}) is a continuous function \(\gamma:[a,b]\to\C\).
\end{defn}
\begin{defn}
An arc is \emph{simple} if it does not cross itself, \emph{closed} if its endpoints meet, and \emph{simple closed} if only its endpoints meet.
\end{defn}
\begin{defn}
An arc is \emph{smooth} if it is continuously differentiable everywhere and its derivative is nonzero everywhere except possibly at its endpoints.
\end{defn}
\begin{defn}
A \emph{contour} is a piecewise smooth arc.
\end{defn}
\begin{defn}
A simple closed contour is \emph{positively oriented} if it is parametrized in the counterclockwise direction, and \emph{negatively oriented} if it is parametrized in the clockwise direction.
\end{defn}
\begin{defn}
A domain is \emph{simply connected} if every simple closed contour lying in the domain encloses only points in the domain. Otherwise it is \emph{multiply connected}.
\end{defn}
\begin{rmk}
Intuitively, a domain is simply connected if it has no holes.
\end{rmk}

\subsection*{Theorems}
\begin{thm}[Properties of contour integrals]
\ 
\begin{enumerate}[itemsep=0pt]
\item[(a)] (Linearity) If \(f\)~and~\(g\) are piecewise continuous on the contour~\(C\) and \(z_0\in\C\), then
\[\int_C f+g=\int_C f+\int_C g\qquad\int_C z_0 f=z_0\int_C f\]
\item[(b)] (Additivity) If \(f\)~is piecewise continuous on the contour \(C_1+C_2\), then
\[\int_{C_1+C_2}f=\int_{C_1}f+\int_{C_2}f\]
\item[(c)] (Boundedness) If \(f\)~is piecewise continuous on the contour~\(C\) and \(\abs{f}\le M\) on~\(C\), then
\[\abs{\int_C f}\le ML\]
where \(L\)~is the length of~\(C\). Such an~\(M\) always exists, and if \(C\)~is parametrized by~\(z(t)\) on~\([a,b]\), then \(L=\int_a^b\abs{z'(t)}\,dt\).
\end{enumerate}
\end{thm}
\begin{proof}[Proof idea]
By definition of the contour integral and results from real analysis.
\end{proof}

\begin{thm}[Fundamental theorem of calculus]
Let \(f\)~be continuous on a domain~\(D\).
\begin{enumerate}[itemsep=0pt]
\item[(a)] If integrals of~\(f\) are independent of path in~\(D\) and \(z_0\in D\), then
\[F(z)=\int_{z_0}^z f\qquad(z\in D)\]
is analytic in~\(D\), and \(F'=f\) in~\(D\).
\item[(b)] If \(F'=f\) in~\(D\), then integrals of~\(f\) are independent of path in~\(D\). In particular, if \(z_0,z_1\in D\) and \(C\)~is any contour in~\(D\) from~\(z_0\) to~\(z_1\), then
\[\int_C f=F(z_1)-F(z_0)\]
\end{enumerate}
\end{thm}
\begin{proof}[Proof idea]
For~(a), use a line segment from~\(z_0\) to~\(z\) and the proof from real analysis.

For~(b), use the definition of the contour integral and the fundamental theorem of calculus from real analysis. In detail, assume without loss of generality that \(C\)~is a smooth arc parametrized by~\(z(t)\) on~\([a,b]\). Then
\[\frac{d}{dt}F(z(t))=F'(z(t))z'(t)=f(z(t))z'(t)\]
Therefore
\begin{equation*}
\int_C f=\int_a^b f(z(t))z'(t)\,dt=F(z(b))-F(z(a))=F(z_1)-F(z_0)\qedhere
\end{equation*}
\end{proof}
\begin{app}
Showing that integration and differentiation are inverse processes, calculating integrals and derivatives, etc.
\end{app}

\begin{cor}
Let \(f\)~be continuous on a domain~\(D\). Then the following are equivalent:
\begin{enumerate}[itemsep=0pt]
\item[(a)] \(f\)~has an antiderivative in~\(D\)
\item[(b)] Integrals of~\(f\) are independent of path in~\(D\).
\item[(c)] Integrals of~\(f\) along closed contours in~\(D\) vanish.
\end{enumerate}
\end{cor}

\begin{thm}[Jordan]
Let \(C\)~be a simple closed contour. Then \(C\)~is the boundary of two distinct domains, an interior which is bounded and an exterior which is unbounded.\footnote{Not proved in~\cite{brown04}.}
\end{thm}
\begin{app}
Cauchy-Goursat, etc.
\end{app}

\begin{thm}[Cauchy-Goursat]
If \(f\)~is analytic on and inside a simple closed contour~\(C\), then
\[\int_C f=0\]
\end{thm}
\begin{proof}[Proof idea]
Let \(R\)~be the region consisting of \(C\)~and its interior. Break the integral over~\(C\) up into a sum of integrals over adjacent squares covering~\(R\). Approximate these integrals by choosing inside each square a sample point at which the derivative of~\(f\) well approximates difference quotients, and thus relates to values of~\(f\) on the square. Show that these integrals can be made arbitrarily small by taking arbitrarily small squares, and hence vanish.

To show that suitable squares and sample points can always be chosen, use the analyticity of~\(f\) in a subdivision argument. By the Jordan curve theorem, \(R\)~can be covered with a single square. If no sample point exists in this square, subdivide it into four smaller squares; if sample points do not exist in each of these squares, subdivide the ones without a sample point; and so on. This process generates a descending tree of squares. If the process never ends, take a limit point in some descending chain. Since \(f\)~is analytic at this point, difference quotients are well approximated nearby. But this includes small enough squares in the chain, so the process must end after all.
\end{proof}
\begin{cor}[Contours inside a contour]
Let \(C\)~be a simple closed contour. Let \(C_0,\ldots,C_k\) be simple closed contours inside~\(C\), of the same orientation as~\(C\), which are pairwise disjoint and whose interiors are pairwise disjoint. Then if \(f\)~is analytic on all these contours and between \(C\)~and the \(C_0,\ldots,C_k\), then
\[\int_C f=\sum_{i=0}^k\int_{C_i}f\]
\end{cor}
\begin{proof}[Proof idea]
Draw polygonal lines connecting the contours \(C,C_0,\ldots,C_k\) in order to express \(\int_C f+\sum\int_{-C_i}f\) as a sum of integrals about simple closed contours.
\end{proof}
\begin{app}
Calculating integrals, deformation of path, Cauchy's residue theorem.
\end{app}
\begin{cor}[Deformation of path]
Let \(C\)~be a simple closed contour, and let \(C_0\)~be a simple closed contour inside~\(C\) of the same orientation. If \(f\)~is analytic on and between \(C\)~and~\(C_0\), then
\[\int_C f=\int_{C_0}f\]
\end{cor}
\begin{app}
Simplifying integrals, Cauchy's integral formula, Laurent series.
\end{app}
\begin{cor}[Analyticity implies antiderivative]
If \(f\)~is analytic on a simply connected domain~\(D\), \(f\)~has an antiderivative on~\(D\).
\end{cor}
\begin{proof}[Proof idea]
Use a stronger version of Cauchy-Goursat which states that whenever \(f\)~is analytic on a simply connected domain, integrals of~\(f\) around \emph{arbitrary} closed contours inside the domain vanish.\footnote{Not proved in~\cite{brown04}.} Then apply the fundamental theorem.
\end{proof}

\begin{thm}[Cauchy integral formula]
If \(f\)~is analytic on and inside a positively oriented simple closed contour~\(C\) and \(z_0\)~is inside~\(C\), then
\[f^{(n)}(z_0)=\frac{n!}{2\pi i}\int_C\frac{f(z)}{(z-z_0)^{n+1}}\,dz\qquad(n\ge0)\]
\end{thm}
\begin{proof}[Proof idea]
Prove case \(n=0\), leaving the general case unproved.\footnote{Not proved in~\cite{brown04} for \(n>2\).}

Use deformation of path to simplify the integral. Let \(C_{\rho}\)~be a positively oriented circle of radius~\(\rho\) about~\(z_0\), inside~\(C\). Then
\begin{align*}
\int_C\frac{f(z)}{z-z_0}\,dz-2\pi if(z_0)&=\int_{C_{\rho}}\frac{f(z)}{z-z_0}\,dz-\int_{C_{\rho}}\frac{f(z_0)}{z-z_0}\,dz\\
	&=\int_{C_{\rho}}\frac{f(z)-f(z_0)}{z-z_0}\,dz
\end{align*}
By continuity of~\(f\) at~\(z_0\), the last integral can be made arbitrarily small by letting \(\rho\to0\), so it vanishes.
\end{proof}
\begin{app}
Relating local behavior of analytic functions and their derivatives at points to global behavior along contours, calculating function and derivative values, calculating integrals, calculating residues, Taylor series, Laurent series, etc.
\end{app}

\begin{cor}[Cauchy inequality]
If \(f\)~is analytic on and inside a positively oriented circle~\(C\) of radius~\(R\) about~\(z_0\), and \(\abs{f}\le M\) on~\(C\), then
\[\abs{f^{(n)}(z_0)}\le\frac{n!M}{R^n}\qquad(n\ge0)\]
\end{cor}

\begin{cor}
If \(f\)~is analytic at a point, then \(f\)~is infinitely analytic at the point.
\end{cor}

\begin{cor}
If \(f=u+iv\) is analytic at a point, then \(u\)~and~\(v\) have continuous partial derivatives of all orders at the point.
\end{cor}

\begin{rmk}
By the above, if \(f\)~is analytic at a point, local behavior of~\(f\) at the point is related to global behavior of~\(f\). Moreover, \(f\)~can be infinitely differentiated and antidifferentiated at the point. In this way, analytic functions are like power series. This is confirmed by the fact that \(f\)~is analytic at a point iff it has a Taylor series expansion there.
\end{rmk}

\begin{thm}[Morera]
Let \(f\)~be continuous on a domain~\(D\). If \(\int_C f=0\) for every closed contour~\(C\) inside~\(D\), then \(f\)~is analytic on~\(D\).
\end{thm}
\begin{proof}[Proof idea]
By the fundamental theorem, \(f\)~has an antiderivative on~\(D\). So \(f\)~is the derivative of an analytic function on~\(D\), hence is analytic on~\(D\).
\end{proof}
\begin{app}
A converse to Cauchy-Goursat (for continuous functions, on simply connected domains).
\end{app}

\begin{thm}[Liouville]
If \(f\)~is bounded and entire, then \(f\)~is constant.
\end{thm}
\begin{proof}[Proof idea]
By Cauchy's inequality, show that \(f'=0\) by letting \(R\to\infty\).
\end{proof}
\begin{app}
Fundamental theorem of algebra.
\end{app}

\begin{thm}[Fundamental theorem of algebra]
If \(P\)~is a nonconstant polynomial, then \(P\)~has a root.
\end{thm}
\begin{proof}[Proof idea]
By Liouville's theorem. If \(P\)~has no root, then \(1/P\)~is bounded and entire, hence constant. But then \(P\)~is constant, which is false.
\end{proof}
\begin{app}
Algebraic closure of~\(\C\), factoring polynomials into linear factors, etc.
\end{app}

\begin{thm}[Maximum modulus principle]
If \(f\)~is nonconstant and analytic on a domain~\(D\), and then \(\abs{f}\)~has no maximum value on~\(D\).
\end{thm}
\begin{proof}[Proof idea]
Use Cauchy's integral formula and integral bounds to argue that the value of~\(\abs{f}\) at a point in~\(D\) is the arithmetic mean of the values of~\(\abs{f}\) on any circle in~\(D\) surrounding the point. So if \(\abs{f}\)~has a local maximum in~\(D\), \(f\)~is constant nearby.

Now if \(\abs{f}\)~has a maximum value on~\(D\), any two points in~\(D\) are connected by a sequence of overlapping neighborhoods on each of which \(f\)~is constant, so \(f\)~is constant on~\(D\).
\end{proof}

\begin{cor}
If \(f\)~is nonconstant and continuous on a closed and bounded region~\(R\), and analytic interior to~\(R\), then \(\abs{f}\)~attains a maximum value on the boundary of~\(R\) but not interior to~\(R\).
\end{cor}
\begin{proof}[Proof idea]
By the extreme value theorem from real analysis, \(\abs{f}\)~attains a maximum value on~\(R\). By the maximum modulus principle, it must occur on the boundary.
\end{proof}

\subsection*{Techniques}
\begin{itemize}[itemsep=0pt]
\item Evaluating contour integrals:
\begin{itemize}[itemsep=0pt]
\item Definition (and results from real analysis).
\item Properties of integrals (linearity, additivity, etc.).
\item Fundamental theorem of calculus.
\item Cauchy-Goursat.
\item Deformation of path.
\item Cauchy's integral formula (and Cauchy's inequality).
\end{itemize}
\item Using subdivision to construct descending chains and take limit points.
\item Relating differentiation and integration using the fundamental theorem of calculus.
\item Relating local properties of functions and their derivatives to global properties using Cauchy's integral formula, Cauchy's inequality, etc.
\item Transferring local properties of functions to global properties using a sequence of overlapping neighborhoods in a connected domain.
\end{itemize}
