%
% Notes on Mathematics
% John Peloquin
%
% Analysis
% Complex Analysis
% Elementary Functions
%
\section{Elementary Functions}
\subsection*{Definitions}
Definitions of real elementary functions assumed.
\begin{defn}
The \emph{exponential function} is given by
\[e^z=e^x(\cos y+i\sin y)\]
where \(z=x+iy\). It is also denoted~\(\exp z\).
\end{defn}
\begin{defn}
The trigonometric functions \(\cos\)~and~\(\sin\) are given by
\[\cos z=\frac{e^{iz}+e^{-iz}}{2}\qquad\sin z=\frac{e^{iz}-e^{-iz}}{2i}\]
The rest of the trigonometric functions are then defined as usual.
\end{defn}
\begin{defn}
The hyperbolic functions \(\cosh\)~and~\(\sinh\) are given by
\[\cosh z=\frac{e^z+e^{-z}}{2}\qquad\sinh z=\frac{e^z-e^{-z}}{2}\]
The rest of the hyperbolic functions are then defined as usual.
\end{defn}
\begin{defn}
For \(z\ne0\), the multivalued \emph{argument function}~\(\arg z\) gives the angles of~\(z\) from polar form. The \emph{principal argument function}~\(\Arg z\) gives the unique angle in \((-\pi,\pi]\).
\end{defn}
\begin{defn}
The multivalued \emph{logarithmic function} is given by
\[\log z=\ln\abs{z}+i\arg z\quad(z\ne0)\]
\end{defn}
\begin{defn}
The multivalued \emph{generalized power function} is given by
\[z^c=\exp(c\log z)\quad(z\ne0)\]
\end{defn}
\begin{defn}
The multivalued \emph{generalized exponential function} is given by
\[c^z=\exp(z\log c)\quad(c\ne0)\]
\end{defn}

\begin{defn}
A \emph{branch} of a multivalued function~\(f\) is a function~\(f^*\) defined on some domain~\(D\) which is analytic on~\(D\) and takes on values of~\(f\) in~\(D\).

A \emph{branch cut} of~\(f\) is a portion of a line or curve used to define a branch of~\(f\).

A \emph{branch point} of~\(f\) is a point common to all branch cuts of~\(f\).
\end{defn}

\begin{defn}
The \emph{principal branch} of the logarithmic [generalized power, generalized exponential] function uses angles in \((-\pi,\pi)\) and is denoted~\(\Log z\) [P.V.~\(z^c\), P.V. \(c^z\)].
\end{defn}
\subsection*{Theorems}
Basic properties of elementary functions assumed.
\begin{rmk}
The following functions are all multivalued without suitable restrictions (e.g., to branches):
\begin{itemize}[itemsep=0pt]
\item \(\arg\) (argument)
\item \(\log\) (logarithm)
\item \(z^c\) (generalized power function, including \(n\)-th root function~\(z^{1/n}\))
\item \(c^z\) (generalized exponential function)
\item Inverse trigonometric functions
\item Inverse hyperbolic functions
\end{itemize}
All of this owes ultimately to the periodicity of angles in the complex plane.

There is sometimes ambiguity. For instance, does \(e^{1/n}\)~denote the single real value~\(\sqrt[n]{e}\), or the set of \(n\)~complex \(n\)-th roots of~\(e\)? The former is conventionally the value of~\(e^z\) at \(z=1/n\), while the latter is the value of \(z^{1/n}\) at \(z=e\). Also, note \(c^z\)~is multivalued in general, while \(e^z\)~is not (\(e^z\)~is obtained from~\(c^z\) by taking the principal branch with \(c=e\)). Ambiguity is resolved by context.
\end{rmk}
