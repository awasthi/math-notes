%
% Notes on Mathematics
% John Peloquin
%
% Analysis
% Complex Analysis
% Applications of Residues
%
\section{Applications of Residues}
\subsection*{Definition}
\begin{defn}
A function is \emph{meromorphic} in a domain if it is analytic throughout the domain except possibly at poles.
\end{defn}
\begin{defn}
If \(f\)~is continuous and nonzero on a positively oriented simple closed contour~\(C\), and \(K\)~is the closed image contour, and \(\wind{C}{f}\)~denotes the change in argument of an image point~\(f(z)\) after \(z\)~makes one oriented traversal of~\(C\), then the \emph{winding number (with respect to the origin)} of~\(K\) is
\[\frac{1}{2\pi}\wind{C}{f}\]
\end{defn}
\begin{rmk}
Intuitively, the winding number of a curve is the number of times it winds around the origin.
\end{rmk}
\subsection*{Theorems}
\begin{thm}[Argument principle]
Let \(C\)~be a positively oriented simple closed contour. If \(f\)~is analytic and nonzero on~\(C\) and meromorphic inside~\(C\), then
\[\frac{1}{2\pi}\wind{C}{f}=Z-P\]
where \(Z\)~is the number of zeros of~\(f\) inside~\(C\) and \(P\)~is the number of poles of~\(f\) inside~\(C\), both counting multiplicity (order).
\end{thm}
\begin{proof}[Proof idea]
By evaluating \(\int_C f'/f\) in two ways, once directly using polar form and again using Cauchy's residue theorem.
\end{proof}
\begin{app}
Calculating winding numbers, counting zeros and poles.
\end{app}

\begin{cor}[Rouch\'e]
If \(f\)~and~\(g\) are analytic on and inside a simple closed contour~\(C\) and \(\abs{f}>\abs{g}\) on~\(C\), then \(f\)~and \(f+g\) have the same number of zeros inside~\(C\).
\end{cor}
\begin{proof}[Proof idea]
By the argument principle. Note \(f\)~and \(f+g\) are nonzero on~\(C\), so
\[Z_f=\frac{1}{2\pi}\wind{C}{f}\qquad Z_{f+g}=\frac{1}{2\pi}\wind{C}{(f+g)}\] 
where \(Z_f\)~and~\(Z_{f+g}\) denote the number of zeros inside~\(C\) of \(f\)~and \(f+g\), respectively. But
\begin{align*}
\wind{C}{(f+g)}&=\wind{C}{f[1+g/f]}\\
	&=\wind{C}{f}+\wind{C}{(1+g/f)}
\end{align*}
But \(\abs{g}/\abs{f}<1\) on~\(C\), so \(\wind{C}{(1+g/f)}=0\).
\end{proof}
\begin{app}
Counting zeros.
\end{app}

\subsection*{Techniques}
\begin{itemize}[itemsep=0pt]
\item Evaluating certain \emph{real} integrals using appropriately constructed contours and Cauchy's residue theorem, including using indented paths for isolated real singularities, etc.
\item Calculating winding numbers by counting zeros and poles, and conversely (argument principle).
\item Counting zeros and poles:
\begin{itemize}[itemsep=0pt]
\item Winding numbers (argument principle).
\item Rouch\'e's theorem.
\end{itemize}
\end{itemize}
